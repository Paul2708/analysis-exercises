\documentclass[parskip=full]{scrartcl}

\usepackage{lmodern}
\usepackage[T1]{fontenc}
\usepackage[utf8]{inputenc}
\usepackage[ngerman]{babel}

\usepackage{amssymb}
\usepackage{amsmath}
\usepackage{amsopn}

\DeclareMathOperator{\grad}{grad}
\DeclareMathOperator{\diff}{\mathop{}\!\mathrm{d}}
\DeclareMathOperator{\vol}{vol}
\DeclareMathOperator{\cp}{CP}

\usepackage{hyperref}

\title{Höhere Mathematik I \& II\\- Analysis für Informatiker}
\author{Paul Hoger}
\date{\today}

\begin{document}

\maketitle

\tableofcontents

\section{Folgen}
\subsection{Grenzwerte}
\begin{displaymath}
  \lim_{n \to \infty} {\sqrt{n^2 + n} - 3n} = - \infty
\end{displaymath}
Idee: 3. Binomische Formel

\begin{displaymath}
  \lim_{n \to \infty} \frac{(n+1)^n \cdot (2n+1)^3}{n^n} = 8e
\end{displaymath}
Idee: Ausdividieren und $\frac{a^n}{b^n} = \left( \frac{a}{b}\right) ^n$ ausnutzen.

\begin{displaymath}
  \lim_{n \to \infty} (\sqrt{3n-2} - \sqrt{3n+1}) = 0
\end{displaymath}
Idee: 3. Binomische Formel

\begin{displaymath}
  \lim_{n \to \infty} \left(1 + \frac{a}{b \cdot n + k}\right)^{cn} = e^{c \cdot \frac{a}{b}}
\end{displaymath}
Idee: Kein Plan. Ist aber so

\begin{displaymath}
  \liminf_{n \to \infty} \left(\left(1 + \frac{2}{n}\right)^n \cdot (-1)^n \right) = -e^2
\end{displaymath}
Idee: Haufüngswerte bestimmen oder Limes reinziehen

\begin{displaymath}
  \lim_{n \to \infty} \left(1 - \frac{1}{n^2}\right)^n = 1
\end{displaymath}
Idee: 3. Binomische Formel und Grenzwertsatz benutzen

\subsection{Rekursive Folgen}
Sei $a_1 := \frac{3}{2}$ und $a_{n+1} = \frac{3}{4-a_n}$.
\begin{enumerate}
    \item Zeigen Sie mit vollständiger Induktion $1 \leq a_{n+1} \leq a_n \leq \frac{3}{2}$.
    \item Zeigen Sie, dass die Folge $a_n$ konvergiert und berechnen Sie ihren Grenzwert.
\end{enumerate}

Antwort:
\begin{enumerate}
    \item
    \begin{enumerate}
        \item Betrachte $1 \leq a_n \leq \frac{3}{2}$ und $a_{n+1} \leq a_n$.
        \item Induktionsanfang ist klar.
        \item Induktionsschritt mit $a_{n+1} = \frac{3}{4 - a_n}$ anfangen (bzw. $a_{n+2} = \frac{3}{4 - a_{n+1}}$) und IV einsetzen
    \end{enumerate}
    \item
    \begin{enumerate}
        \item Mit 1. folgt, dass die Folge monton fallend und beschränkt ist.
        Daher konvergiert sie auch.
        \item Für den Grenzwert $a := \lim_{n \to \infty} a_n$ gilt:
        \begin{displaymath}
          a = \frac{3}{4 - a}
          \iff (a-3)(a-1) = 0
          \iff a = 1.
        \end{displaymath}
    \end{enumerate}
\end{enumerate}

Sei $a \in (-1,1)$, $b \in \mathbb{R}$, $x_0 \in \mathbb{R}$ beliebig. Für $n \in \mathbb{N}_0$ definiere
\begin{displaymath}
  x_{n+1} := ax_n + b.
\end{displaymath}
Zeigen Sie:
\begin{enumerate}
    \item Es existiert genau ein $x \in \mathbb{R}$ mit $x = ax+b$. Bestimmen Sie dieses.
    \item Für alle $n \in \mathbb{N}$ gilt $x_n - \frac{b}{1-a} = a^n\left(x_0 - \frac{b}{1-a}\right)$.
    \item $\lim_{n \to \infty} x_n = \frac{b}{1-a}$.
\end{enumerate}

Antwort:
\begin{enumerate}
    \item
    \begin{enumerate}
        \item $x$ ausklammern, durch $1-a$ teilen.
        \item Mit den Äquivalenzumformungen folgt die Eindeutigkeit und die Existenz.
    \end{enumerate}
    \item
    \begin{enumerate}
        \item Mithilfe vollständiger Induktion.
    \end{enumerate}
    \item
    \begin{enumerate}
        \item Benutze den Satz: $|a_n - a| \leq b_n$ und $b_n \to 0$, so gilt $a_n \to a$.
        \item Betrachte $|x_n - \frac{b}{1-a}|$ und benutze die 2.
        \item Mit $|a| < 1$ folgt die Behauptung.
    \end{enumerate}
\end{enumerate}

\subsection{Wahr oder falsch?}
Besitzt eine Folge reeller Zahlen genau einen Häufungspunkt in $\mathbb{R}$, so ist diese Folge konvergent.\\
Falsch.\\
Betrachte $a_n = (0,1,0,2,0,3,\dots)$. $0$ ist offensichtlich ein Häufungspunkt, $a_n$ ist aber nichtmals beschränkt.

Seien $a_n$, $b_n$ beschränkte Folgen reeller Zahlen.
\begin{displaymath}
  \liminf_{n \to \infty} (a_n - b_n) = \liminf_{n \to \infty} a_n - \liminf_{n \to \infty} b_n
\end{displaymath}
Falsch.\\
Gegenbeispiel: $a_n = 1$, $b_n = (-1)^n$ oder $a_n = (0,1,0,1,\dots)$, $b_n = (1,0,1,0,\dots)$.

Sei $a_n$ monoton und nach oben beschränkt.
Dann ist $a_n$ konvergent.\\
Falsch.\\
Die Folge kann auch monton fallend sein.

Sei $a_n$ konvergent mit $a := \lim_{n \to \infty} a_n$.
Dann konvergiert jede Teilfolge gegen $a$.\\
Wahr.\\
Bew.: $\varepsilon$-Umgebung

Eine beschränkte Folge hat einen Häufungswert.\\
Wahr.\\
Bew.: Bolzano-Weierstraß.

Jede reelle Zahlenfolge enthält eine monotone Teilfolge.\\
Wahr.\\
Bew.: Mit Indizes spielen und Definition von niedrig ($m$ heißt niedrig $\iff \forall n \geq m : a_n \geq a_m$).

Sei $a_n$ eine beschränkte Folge.
Dann hat $a_n$ eine monoton fallende Teilfolge.\\
Falsch.\\
Die Teilfolge kann auch monoton steigend sein.

\subsection{Häufungswerte}
Bestimmen Sie die Menge der Häufungswerte der Folge $a_n := e^{(-1)^{n}n-n} \sin\left(\frac{\pi n}{2}\right)$.\\
Antwort:
Für alle $k \in \mathbb{N}$ gilt:
\begin{align*}
  a_{2k} &= e^{(-1)^{2k}2k-2k}\sin\left(\frac{2k\pi}{2}\right) = e^0 \sin(k\pi) = 0\\
  a_{2k+1} &= e^{(-1)^{2k+1}(2k+1)-(2k+1)}\sin\left(\frac{(2k+1)\pi}{2}\right) = e^{-4k-2} \sin\left(\frac{(2k+1)\pi}{2}\right) \to 0 (k \to \infty).
\end{align*}
Da $e^{-4k-2}$ gegen $0$ geht und der Sinus beschränkt ist.

Bestimmen Sie die Menge der Häufungswerte der Folge $a_n := \frac{\left(1+\sin\left(\frac{\pi}{2}n\right)\right)^n}{2}$.\\
Antwort: Für alle $k \in \mathbb{N}$ gilt:
\begin{align*}
  a_{2k} &= \frac{1}{2} \left(1+\sin\left(\frac{\pi}{2}2k\right)\right)^{2k} = \frac{1}{2} \left(1+\sin\left(\pi k\right)\right)^{2k} = \frac{1}{2} (1 + 0)^{2k} = \frac{1}{2}\\
  a_{4k-1} &= \frac{1}{2} \left(1+\sin\left(\frac{\pi}{2}(4k-1)\right)\right)^{4k-1} = \frac{1}{2} (1 - 1)^{4k-1} = 0\\
  a_{4k+1} &= \frac{1}{2} \left(1+\sin\left(\frac{\pi}{2}(4k+1)\right)\right)^{4k+1} = \frac{1}{2} (1 + 1)^{4k+1} \to \infty \text{ } (k \to \infty) 
\end{align*}
Beachte, dass jede Teilfolge betrachtet wurde.

\subsection{Beziehungen}
Sei $a_n$ eine Folge.
\begin{displaymath}
  a_n \to 0 \text{ für } n \to \infty \text{. } \fbox{\rule{1in}{0pt}\rule[-0.5ex]{0pt}{4ex}} \text{ $a_n$ ist beschränkt.}
\end{displaymath}
Mögliche Beziehungen:
$\implies$, $\Leftarrow$, $\iff$, k.B.\\
Antwort: $\implies$

\section{Unendliche Reihen}
\subsection{Reihenwerte}
\begin{displaymath}
  \sum_{k = 0}^{\infty} \frac{(-1)^k}{2^k} = \frac{2}{3}
\end{displaymath}
Idee: Potenzgesetz anwenden und als geometrische Reihe auffassen.

\begin{displaymath}
  \sum_{k = 1}^{\infty} \left(\frac{1}{k+1}-\frac{1}{k}\right) = -1
\end{displaymath}
Idee: Teleskopsumme

\begin{displaymath}
  \sum_{k = 0}^{\infty} \frac{(-1)^k}{3^k} = \frac{3}{4}
\end{displaymath}
Idee: Potenzgesetz anwenden und als geometrische Reihe auffassen.

\begin{displaymath}
  \sum_{n = 1}^{\infty} \frac{2^n}{3^{n+1}} = \frac{2}{3}
\end{displaymath}
Idee: Ausklammern, Index-Shift, geometrische Reihe

\subsection{Wahr oder Falsch?}
Sei $a_n$ eine beschränkte Folge reeller Zahlen.
Falls $a_n \cdot a_{n+1} < 0$ für alle $n \in \mathbb{N}$, dann konvergiert $\sum\limits_{n = 1}^{\infty} a_n$.\\
Falsch.\\
Betrachte das Gegenbeispiel $a_n = (-1)^n$.

Sei $a_n$ eine Folge mit $\sqrt[n]{|a_n|} < 1$ für alle $n \in \mathbb{N}$. Dann konvergiert die Reihe $\sum\limits_{n = 1}^{\infty} a_n$.\\
Falsch.\\
Falsche Anwendung des Wurzelkriteriums. Der Limes Superior muss $< 1$ sein, nicht die Folgenglieder.

Sind $a_n \in \mathbb{R}$, $n \in \mathbb{N}$, und gilt $|a_n| \leq \frac{1}{n}$ für alle $n \in \mathbb{N}$, dann ist die Reihe $\sum_{n = 1}^{\infty} (-1)^n a_n$ konvergent.\\
Falsch.\\
Betrachte $a_n := \left((-1)^n \frac{1}{n}\right)$. Die Folge erfüllt die Bedingung, aber $\sum_{n = 1}^{\infty} (-1)^n a_n = \sum_{n = 1}^{\infty} \frac{1}{n}$ ist divergent.

Es gibt absolut konvergente Reihen, die nicht konvergieren.\\
Falsch.\\
Nach der VL gilt
\begin{displaymath}
  \sum_{n = 1}^{\infty} a_n \text{ ist absolut konvergent. } \implies \sum_{n = 1}^{\infty} a_n \text{ ist konvergent. }
\end{displaymath}

Es gibt konvergente Reihen $\sum_{n = 0}^{\infty} a_n$ derart, dass das Chauchy-Produkt der Reihe mit sich selbst divergiert.\\
Wahr.\\
Betrachte die Reihe $\sum_{k = 1}^{\infty} \frac{(-1)^{k+1}}{\sqrt{k}}$.
Mit dem Leibniz-Kriterium folgt die Konvergenz.
Das Chauchy-Produkt der Reihe mit sich selbst ist $\sum_{m = 1}^{\infty} c_m$ mit $c_m = (-1)^m \sum_{j = 1}^{m - 1} \frac{1}{\sqrt{j} \sqrt{m-j}}$.
Da $|c_m| \geq 1$ für alle $m \geq 2$ ist $\sum_{m = 1}^{\infty} c_m$ also divergent.

\subsection{Beziehungen}
Sei $a_n$ eine reelle Folge.
\begin{displaymath} 
  \sum_{n = 1}^{\infty} a_n^2 \text{ konvergiert. } \fbox{\rule{1in}{0pt}\rule[-0.5ex]{0pt}{4ex}} \text{ } \sum_{n = 1}^{\infty} a_n \text{ konvergiert.}
\end{displaymath}
Antwort: keine Beziehung

\begin{displaymath}
  \sum_{n = 1}^{\infty} a_n^2 \text{ konvergiert. } \fbox{\rule{1in}{0pt}\rule[-0.5ex]{0pt}{4ex}} \text{ } \sum_{n = 1}^{\infty} a_n \text{ konvergiert absolut.}
\end{displaymath}
Antwort: $\Leftarrow$

\begin{displaymath}
  \sum_{n = 1}^{\infty} a_n \text{ konvergiert. } \fbox{\rule{1in}{0pt}\rule[-0.5ex]{0pt}{4ex}} \text{ } \limsup_{n \to \infty} \sqrt[n]{|a_n|} \leq 1
\end{displaymath}
Antwort: $\Rightarrow$

\begin{displaymath}
  \sum_{n = 1}^{\infty} a_n \text{ konvergiert absolut. } \fbox{\rule{1in}{0pt}\rule[-0.5ex]{0pt}{4ex}} \text{ } \limsup_{n \to \infty} \sqrt[n]{|a_n|} \leq 1
\end{displaymath}
Antwort: $\Rightarrow$

\begin{displaymath}
  \sum_{n = 1}^{\infty} a_n \text{ konvergiert absolut. } \fbox{\rule{1in}{0pt}\rule[-0.5ex]{0pt}{4ex}} \text{ } a_n \to 0 \text{ für } n \to \infty \text{.}
\end{displaymath}
Antwort: $\Rightarrow$

\begin{displaymath}
  a_n \to 0 \text{ für } n \to \infty \text{.} \fbox{\rule{1in}{0pt}\rule[-0.5ex]{0pt}{4ex}} \text{ } a_n \text{ ist beschränkt.}
\end{displaymath}
Antwort: $\Rightarrow$

\section{Potenzreihen}
\subsection{Konvergenzradius}
\begin{displaymath}
  \sum_{n = 1}^{\infty} \frac{(n!)^2}{(2n)!}x^n \implies r = 4
\end{displaymath}
Idee: Quotientenkriterium, Fakultät kürzen

\begin{displaymath}
  \sum_{n = 1}^{\infty} \left(1 - \frac{1}{n}\right)^{n^2} x^{8n} \implies r = \sqrt[8]{e}
\end{displaymath}
Idee: Wurzel-Kriterium für Reihen

\begin{displaymath}
  \sum_{n = 1}^{\infty} \left(1 - \frac{1}{n}\right)^{n^2}x^n \implies r = e
\end{displaymath}
Idee: Wurzelkriterium für Reihen

\begin{displaymath}
  \sum_{n = 1}^{\infty} \frac{n^n}{n!} (x-1)^n \implies r = e^{-1}
\end{displaymath}
Idee: Quotientenkriterium für Potenzreihen und invertieren

\begin{displaymath}
  \sum_{n = 0}^{\infty} (-1)^n \frac{n!}{2^{n!}} x^n \implies r = \infty
\end{displaymath}
Idee: Quotientenkriterium für Potenreihen

\begin{displaymath}
  \sum_{n = 1}^{\infty} \left(1 + \frac{2}{n}\right)^{n^2} x^n \implies r = e^{-2}
\end{displaymath}
Idee: Definition Konvergenzradius

\subsection{Konvergenzbereich}
\begin{displaymath}
  \sum_{n = 1}^{\infty} \frac{(-1)^n(2x+1)^n}{n} \text{ konvergiert } \iff x \in (-1, 0]
\end{displaymath}
Idee: Wurzelkriterium für Reihen, Ränder checken (Leibniz-Kriterium und harmonische Reihe)

\begin{displaymath}
  \sum_{n = 1}^{\infty} \left(1+\frac{1}{2}+\dots+\frac{1}{n}\right)x^n \text{ konvergiert } \iff |x| < 1
\end{displaymath}
Idee: Konvergenzradius über Sandwich-Theorem ($1 \leq a_n \leq n$), Ränder checken (keine Nullfolgen)

\begin{displaymath}
  \sum_{n = 1}^{\infty} \frac{1}{\sqrt{n}} x^n \text{ konvergiert. } \iff x \in [-1, 1)
\end{displaymath}
Idee: Konvergenzradius bestimmen ($\sqrt{n}$ mit Sandwich), Ränder checken (Minorantenkriterium und Leibniz-Kriterium)

\begin{displaymath}
  \sum_{n = 1}^{\infty} \frac{1}{n!} (x+3)^{2n} \text{ konvergiert } \iff x \in (-\infty, \infty)
\end{displaymath}
Idee: Quotientenkriterium für Reihen

\begin{displaymath}
  \sum_{n = 1}^{\infty} \frac{2n}{4+n^3}(x+1)^{2n} \text{ konvergiert } \iff x \in [-2, 0]
\end{displaymath}
Idee: Substituiere $y := (x+1)^n$ und erhalte $\sum_{n = 1}^{\infty} \frac{2n}{4 + n^3}y^n$, Ränder checken (Majoranten-Kriterium)

\begin{displaymath}
  \sum_{n = 1}^{\infty} \frac{(-1)^n}{5n} x^n \text{ konvergiert } \iff x \in (-1, 1]
\end{displaymath}
Idee: Quotientenkriterium für Potenzreihen, Ränder checken (Leibniz, Minoranten-Kriterium)

\begin{displaymath}
  \sum_{k = 1}^{\infty} \frac{1}{\left(1 + \frac{1}{k}\right)^k} x^k \text{ konvergiert } \iff x \in (-1,1)
\end{displaymath}
Idee: Konvergenzradius bestimmen (Def. oder WK für Reihen), Ränder checken (keine Nullfolgen)

\begin{displaymath}
  \sum_{n = 1}^{\infty} \frac{x^n}{(2n)!} \text{ konvergiert } \iff x \in (-\infty, \infty)
\end{displaymath}
Idee: Quotientenkriterium für Potenzreihen

\begin{displaymath}
  \sum_{n = 1}^{\infty} \left(\frac{1}{2} + \frac{1}{2n}\right)^n x^n \text{ konvergiert } \iff x \in (-2,2)
\end{displaymath}
Idee: Konvergenzradius bestimmen, Ränder checken (keine Nullfolge)

\begin{displaymath}
  \sum_{n = 1}^{\infty} \frac{(x+1)^n}{3^n} \text{ konvergiert } \iff x \in (-4, 2)
\end{displaymath} 
Idee: WK für Reihen, Ränder checken (Divergenz)

\begin{displaymath}
  \sum_{n = 1}^{\infty} \frac{(-1)^n x^{2n}}{3n+1} \text{ konvergiert } \iff x \in [-1,1] \text{ oder (-1,1] idk}
\end{displaymath}
Idee: Substituieren (?)

\begin{displaymath}
  \sum_{n = 1}^{\infty} (\sqrt{n^2-3n} - n) x^n \text{ konvergiert } \iff x \in (-1,1)
\end{displaymath}
Idee: Fallunterscheidung: Für $x \geq 1$ ist es keine Nullfolge. Für $x < 1$: Aus der Konvergenz folgt die Beschränktheit. Mit dem Sandwichkriterium (geometrische Reihe) divergiert es.

\begin{displaymath}
  \sum_{n = 1}^{\infty} \left(\frac{1}{3} + \frac{1}{2n}\right)^{n^3} x^{2n} \text{ konvergiert } \iff \text{ ???}
\end{displaymath}
Idee: ???

\section{q-adische Entwicklung}
Sei $0,121212\dots$ die $3$-adische Entwicklung einer Zahl $a \in \mathbb{R}$. Bestimmen Sie $m, n \in \mathbb{N}$ mit $a = \frac{m}{n}$.\\
Lösung: $a = \frac{5}{8}$\\
Idee: Ausgehend von der q-adschen Darstellung kann man zwei Summen betrachten, Index-Shifts, ausklammern, geometrische Reihe 
\section{Grenzwerte bei Funktionen}
\begin{displaymath}
  \lim_{x \to \infty} \frac{\log\left(1 + \frac{1}{x}\right)}{\frac{1}{x} + e^{-x}} = 1
\end{displaymath}
Idee: l'Hospital (bis zum Umfallen)

\begin{displaymath}
  \lim_{x \to \infty} \frac{\sin(2x)}{e^{3x} - 1} = \frac{2}{3}
\end{displaymath}
Idee: l'Hospital

\begin{displaymath}
  \lim_{x \to \infty} \frac{e^{\sin(x)}\cos(x)}{1+x} = 0
\end{displaymath}
Idee: Sandwich-Lemma

\begin{displaymath}
  \lim_{x \to \infty} \frac{\tan(x)-x}{x^3} = \frac{1}{3}
\end{displaymath}
Idee: l'Hospital

\begin{displaymath}
  \lim_{x \to \infty} \frac{x-\sin(x)}{x + \sin(x)} = 1
\end{displaymath}
Idee: Sandwich-Lemma oder alternativ mit $x$ kürzen und $\sin(x)$ beschränkt

\begin{displaymath}
  \lim_{x \to \infty} \frac{5x^3 +\sin(x)}{x(x-2)(2x+3)\cos\left(\frac{1}{x}\right)} = \frac{5}{2}
\end{displaymath}
Idee: Sandwich-Lemma

\begin{displaymath}
  \lim_{x \to 0} \frac{1 - \cos(x)}{x^2} = \frac{1}{2}
\end{displaymath}
Idee: l'Hospital

\begin{displaymath}
  \lim_{x \to 0} \frac{\sin(x) - x\cos(x)}{x\sin(x)+\cos(x) - 1} = 0
\end{displaymath}
Idee: l'Hospital

\begin{displaymath}
  \lim_{x \to 0} \frac{3 \tan(x)}{x} = 3
\end{displaymath}
Idee: l'Hospital

\begin{displaymath}
  \lim_{x \to \infty} \left(2+5x\right)^{\frac{1}{4x}} = 1
\end{displaymath}
Idee: $a = e^{\ln a}$, l'Hospital und Grenzwertsätze benutzen

\begin{displaymath}
  \lim_{x \to 0} \frac{\cos(x) - \sqrt[3]{\cos(x)}}{\sin^2(x)} = -\frac{1}{3}
\end{displaymath}
Idee: l'Hospital

\begin{displaymath}
  \lim_{x \to \frac{\pi}{4}} \frac{\log(\tan(x))}{1 - \frac{1}{\tan(x)}} = 1
\end{displaymath}
Idee: $\tan\left(\frac{\pi}{4}\right) = 0$ und $(\tan(x))' = \frac{1}{\cos^2(x)} = 1 + \tan^2(x)$ und l'Hospital

\begin{displaymath}
  \lim_{x \to \infty} \frac{x^5}{\sinh(x) - x - \frac{x^3}{6}} = 120
\end{displaymath}
Idee: $\sinh(x) = e^x - e^{-x}$ und l'Hospital

\begin{displaymath}
  \lim_{x \to 0} \frac{1}{x^2} \left(\sqrt{1 + x} - 1 - \frac{x}{2}\right) = -\frac{1}{8}
\end{displaymath}
Idee: Taylor-Entwicklung oder l'Hospital

\begin{displaymath}
  \lim_{x \to 0} \frac{\cos(x)}{\sin(x)} \text{ ex. nicht}
\end{displaymath}
Idee: Betrachte linksseitigen und rechtsseitigen Grenzwert

\section{Stetigkeit}
\subsection{Lösung einer Gleichung}
Zeigen Sie, dass die Gleichung
\begin{displaymath}
  (1+x^3)\sqrt{x} = 1
\end{displaymath}
genau eine Lösung $x > 0$ hat. \textit{Hinweis: Zwischenwertsatz und Monotonie.}\\
Lösung: Eindeutigkeit: Produkt zweier streng monotonen Funktionen, die strikt positiv sind.
Damit folgt die Injektivität.\\
Existenz: Zwischenwertsatz für $[0,1]$.

\subsection{Wahr oder Falsch?}
Sei $x_n$ eine reelle Zahlenfolge und $f,g : \mathbb{R} \to \mathbb{R}$ Funktionen.\\
Ist $f$ eine monoton fallende Funktion und $x_n$ monoton fallend, so ist $(f(x_n))_{n \in \mathbb{N}}$ monoton fallend.\\
Falsch.\\
Kein Gegenbeispiel gefunden :(.

Für stetiges $f$ und $a > 0$ gibt es ein $x \in [-a,a]$ so, dass gilt: $f(x) = \sup\{f(y) : y \in [-a,a]\}$.\\
Wahr.\\
Folgt mit dem Zwischenwertsatz.

Sei $f : \mathbb{R} \to \mathbb{R}$ beschränkt.
Dann exisitert $\max_{x \in \mathbb{R}} f(x)$.\\
Falsch.\\
Gegenbeispiel: Betrachte beschränkte Funktion, die nur ein Supremum, aber kein Maximum bestizt.

Ist $f \in C([a,b])$ und $f(a)f(b) < 0$, so existiert ein $x_0 \in [a,b]$, sodass $f(x_0) = 0$.\\
Wahr.\\
Nullstellensatz von Bolanzo.

Ist $f : \mathbb{R} \to \mathbb{R}$ eine bijektive und streng monoton wachsende Funktion.
Dann ist auch $f^{-1}$ streng monoton wachsend.\\
Wahr.\\
Bemerkung aus der Vorlesung.

Eine Teilmenge von $\mathbb{R}$, die nicht offen ist, ist abgeschlossen.\\
Falsch.\\
Die Menge $(0,1]$ ist weder offen noch abgeschlossen.

Seien $f,g \in C(\mathbb{R})$ gleichmäßig stetig.
Dann ist $fg$ gleichmäßig stetig.\\
Falsch.\\
Wähle  $f,g \in \mathbb{R} \to \mathbb{R}, f(x) := x, g(x) := x$.
Dann sind $f$ und $g$ gleichmäßig stetig auf $\mathbb{R}$, aber $(fg)(x) = x^2$, also ist $fg$ nach Vorlesung nicht gleichmäßig stetig auf $\mathbb{R}$.

Es sei $f \in C([0,1])$. Dann ist $f$ gleichmäßig stetig auf $[0,1]$.\\
Wahr.\\
Satz aus der VL.

Es sei $f \in C([0,1])$. Dann ist $f$ Lipschitz-stetig auf $[0,1]$.\\
Wahr.\\
$f(x) = \sqrt{x}$ ist auf $[0,1]$ stetig, aber nicht Lipschitz-stetig.

\subsection{Beziehungen}
Es sei $f : \mathbb{R} \to \mathbb{R}$ eine Funktion.

\begin{displaymath}
  f \text{ ist stetig. } \fbox{\rule{1in}{0pt}\rule[-0.5ex]{0pt}{4ex}} \text{ } f \text{ ist gleichmäßig stetig.}
\end{displaymath}
Antwort: $\Leftarrow$

\begin{displaymath}
  f \text{ ist Lipschitz-stetig. } \fbox{\rule{1in}{0pt}\rule[-0.5ex]{0pt}{4ex}} \text{ } f \text{ ist gleichmäßig stetig.}
\end{displaymath}
Antwort: $\Rightarrow$

\begin{displaymath}
  f \text{ ist gleichmäßig stetig. } \fbox{\rule{1in}{0pt}\rule[-0.5ex]{0pt}{4ex}} \text{ } f \text{ ist beschränkt.}
\end{displaymath}
Antwort: k.B.

\section{Funktionsfolgen und -reihen}
\subsection{Grenzfunktion und gleichmäßige Konvergenz}
Sei die Funktionsfolge $f_n$ definiert durch
\begin{displaymath}
  f_n : \mathbb{R} \to \mathbb{R} \text{ mit } f_n(x)=
  \begin{cases}
    1        & \text{für } x < 0\\
    \cos(nx) & \text{für } 0 \leq x \leq \frac{\pi}{n}\\
    -1       &\text{für} x > \frac{\pi}{n} 
  \end{cases}
\end{displaymath}
Berechnen Sie die punktweise Grenzfunktion und bestimmen Sie, ob die Funktionsfolge auch gleichmäßig konvergiert.\\
Idee: Fallunterscheidung $x < 0$, $x = 0$ und $x > 0$ (betrachte hier, dass $\frac{\pi}{n}$ eine Nullfolge ist), Stetigkeit der Funktionen prüfen.\\
Lösung: $f(x) = \begin{cases} 1 & \text{für } x \leq 0\\ -1 & \text{für } x > 0\end{cases}$ und keine gleichmäßige Konvergenz.

Untersuchen Sie die Funktionenfolge $f_n : \mathbb{R} \to \mathbb{R}, f_n(x) = n \cdot \sin\left(\frac{x}{n}\right)$ auf punktweise und gleichmäßige Konvergenz.\\
Idee: Fallunterscheidung $x = 0$ und $x \neq 0$ (hier umformen auf $\frac{\sin(y)}{y} \xrightarrow{y \to \infty}$ 1).
Betrachte Folge $x_n = n\pi$ und die Differenz beider Funktionen (oder, dass $f$ nicht beschränkt ist)\\
Lösung: Punktweise konvergent mit $f : \mathbb{R} \to \mathbb{R}, f(x) = x$. Nicht gleichmäig konvergent.

Für $n \in \mathbb{N}$ betrachten wir die Funktion
\begin{displaymath}
  f_n: [0,1] \to \mathbb{R}; x \mapsto n^2x(1-x)^n.
\end{displaymath}
Zeigen Sie, dass die Folge $f_n$ auf ganz $[0,1]$ punktweise gegen $f : [0,1] \to \mathbb{R}$ konvergiert und bestimmen Sie die Grenzfunktion.
Konvergiert die Reihe $f_n$ sogar gleichmäßig auf $[0,1]$?\\
Idee: Nutze $\lim_{n \to \infty} n^m \cdot q^n = 0$ mit $q \in [0,1)$. Und teste die Folge $x_n = \frac{1}{n}$.\\
Lösung: $f(x) = 0$ und $f_n$ konvergiert nicht gleichmäßig

Betrachten Sie die Funktionsfolge $f_n:[0,1] \to \mathbb{R}$ definiert durch
\begin{displaymath}
  f(x) := 
  \begin{cases}
    2nx,& \text{falls } x \in [0,\frac{1}{2n}),\\
    2 - 2nx,& \text{falls } x \in [\frac{1}{2n}, \frac{1}{n}],\\
    0,& \text{falls } x \in (\frac{1}{n}, 1].
  \end{cases}
\end{displaymath}
Prüfen Sie die Funktionsfolge auf punktweise und gleichmäßige Konvergenz.\\
Idee: Fallunterscheidung $x = 0$, $x \neq 0$ (hier gilt $\frac{1}{n} < x$ für ffa $n \in \mathbb{N}$.
Betrachte die Folge $x_n = \frac{1}{2n}$.\\
Lösung: $f(x) = 0$ und keine gleichmäßige Konvergenz.

Konvergiert die Funktionenreihe $\sum_{k = 1}^{\infty} \frac{4 \sin(kx)}{k^2}$ gleichmäßig auf $\mathbb{R}$?\\
Idee: Folge abschätzen. Zeigen, dass die abgeschätzte Reihe konvergiert. $\implies$ Kriterium von Weierstraß\\
Lösung: Ja

Untersuchen Sie die Funktionenfolge $f_n$ definiert durch
\begin{displaymath}
  f_n : \mathbb{R} \to \mathbb{R}, f_n(x) := \cos\left(\frac{x}{n}\right)
\end{displaymath}
auf punktweise und gleichmäßige Konvergenz.\\
Idee: Stetigkeit des Cosinus, betrachte $x_n = \frac{\pi n}{2}$\\
Lösung: $f(x) = 0$ und nicht gleichmäßig Konvergenz.

\subsection{Wahr oder Falsch?}
Für alle $n \in \mathbb{N}$ seien $f_n : [0, \infty) \to \mathbb{R}$ stetige Funktionen und es gelte $f_n \xrightarrow{n \to \infty} f$ gleichmäßig auf $[0,\infty)$. Dann ist $f$ stetig.\\
Wahr.\\
Satz aus der VL.

Wenn $x_n$ divergiert, so divergiert auch $(f(x_n))_{n \in \mathbb{N}}$.\\
Falsch.\\
Betrachte $f(x) = \frac{1}{x}$.

Sei $I \subseteq \mathbb{R}$ ein Intervall, seien $f_n, f, g : I \to \mathbb{R}$ Funktionen und sei $g$ beschränkt.
Falls $f_n \to f$ gleichmäßig auf $I$ für $n \to \infty$, dann gilt $f_n \cdot g \to f \cdot g$ gleichmäßig auf $I$ für $n \to \infty$.\\
Wahr.

\section{Differentialrechnung}
\subsection{Ungleichungen}
\begin{displaymath}
  x \log(x) - y \log(y) \leq (x - y)(1 + \log(x))
\end{displaymath}
Idee: MWS mit $f(x) = x \log(x)$, Abschätzung möglich, da $f'$ monoton wachsend ist

Zeigen Sie, dass $f : [1, \infty) \to \mathbb{R}, f(x) = \log(x)$ Lipschitz-stetig ist.\\
Idee: MWS anwenden für $x, y \in [1, \infty), x < y, \xi \in (x,y), f(x) = \log(x)$\\
Lösung: $L = 1$

Sei $f:[1, \infty) \to \mathbb{R}, f(x) = \log(x)$.
Zeigen Sie:
\begin{displaymath}
  1 = \min\{L \in \mathbb{R}: |f(x) - f(y)| \leq L|x-y|, (x,y \in [1,\infty))\}
\end{displaymath}
Idee: Widerspruchsbeweis: $\alpha = \min\{\dots\}$ und $x,y \to 1$ laufen lassen

Zeigen Sie für alle $x,y \in \mathbb{R}$:
\begin{displaymath}
  |\cos(x)\sin^4(x) - \cos(y)\sin^4(y)| \leq 5|x-y|
\end{displaymath}
Idee: MWS für $f(x) = \cos(x)\sin^4(x)$ und Ableitung abschätzen

Zeigen Sie, dass $f(x) = \sin^2(x)$ $(x \in \mathbb{R})$ Lipschitz-stetig ist.\\
Idee: MWS mit $f(x) = \sin^2(x)$\\
Lösung: $L = 2$

Zeigen Sie:
\begin{displaymath}
  |\cos(e^x) - \cos(e^y)| \leq |x-y|
\end{displaymath}
für $x,y \leq 0$.\\
Idee: MWS mit $f(x) = \cos(e^x)$ und $f'(x) \leq 1$ zeigen

Sei $x > 0$.
Zeigen Sie mithilfe des Mittelwertsatzes
\begin{displaymath}
  \frac{1}{2\sqrt{x + 1}} \leq \sqrt{x + 1} - \sqrt{x} \leq \frac{1}{2\sqrt{x}}.
\end{displaymath}
Idee: MWS mit $f(t) = \sqrt{t}$ auf dem Intervall $(x, x+1)$

\subsection{Ableitungen}
Sei $f(x) = x^x$ für $x \in (0,\infty)$.
Bestimmen Sie $f'(e)$.\\
Idee: $a = e^{\log(a)}$\\
Lösung: $2e^e$

Berechnen Sie die Ableitung der Funktion $g(x) = x^{\sin(x)}$ für $x \in (0,\infty)$.\\
Lösung: $g'(x) = x^{\sin(x)} \left(\cos(x)\log(x) + \frac{\sin(x)}{x}\right)$

Sei $f : (-\infty, 0) \to \mathbb{R}$; $f(x) = e^{\log(-x) + - \sqrt{1 + x^2}}$.
Berechnen Sie $f'(-1)$.\\
Lösung: $f'(-1) = e^{-\sqrt{2}}\left(\frac{1}{\sqrt{2}} - 1\right)$

Die Funktion $f:(0,\infty) \to \mathbb{R}$ sei gegeben durch $f(x) := x^{(x^2)}$.
Berechnen Sie für $x > 0$ die Ableitung $f'(x)$.\\
Lösung: $f'(x) = x^{(x^2 + 1)} (2\log(x) + 1)$

\subsection{Differenzierbarkeit prüfen}
Untersuchen Sie die Funktion
\begin{displaymath}
  f:\mathbb{R} \to \mathbb{R}, f(x) :=
  \begin{cases}
    \sin(x)\cos\left(\frac{1}{x}\right),& x \neq 0,\\
    0,& x = 0
  \end{cases}
\end{displaymath}
in jedem Punkt $x \in \mathbb{R}$ auf Differenzierbarkeit.
Beweisen Sie Ihre Aussagen.\\
Idee: Komposition von diff'baren Funktionen. Zeigen, dass der GW nicht existiert\\
Lösung: Nur für $x \neq 0$ differenzierbar.

\subsection{Umkehrfunktion}
Sei $f: [0,\infty) \to [0,\infty)$ definiert durch $f(x) = (x^2 + x)\sqrt{x}$.
Zeigen Sie, dass $f$ bijektiv ist und berechnen Sie $(f^{-1})'(2)$.\\
Idee:
\begin{enumerate}
    \item $f$ ist als Kompoisiton diff'barer Funktionen wieder diff'bar
    \item Ableitung $> 0$ zeigen, also ist $f$ streng monoton wachsend
    \item Damit (und da $f$ stetig ist), folgt $f$ ist injektiv
    \item Mit $f(0) = 0$ und $f(x) \to \infty (x \to \infty)$ folgt mit dem ZWS, dass $f([0,\infty)) = [0,\infty)$, also surjektiv ist.
    \item Damit folgt die Bijektivität.
    \item Mit $f(1) = 2$ folgt $(f^{-1})'(2) = \frac{1}{f'(1)} = \frac{1}{4}$
\end{enumerate}

\subsection{Wahr oder Falsch?}
Sei $\emptyset \neq D \subseteq \mathbb{R}$ offen und $f:D \to \mathbb{R}$ differenzierbar mit $f'(x) = 0$ für alle $x \in D$. 
Dann existiert ein $c \in \mathbb{R}$ mit $f'(x) = c$ für alle $x \in D$.\\
$D$ ist kein Intervall, sondern eine offene Menge.

Aus $g(x) \leq f(x)$ für alle $x \in \mathbb{R}$ folgt $g'(x) \leq f'(x)$ für alle $x \in \mathbb{R}$.\\
Falsch.

Sei $f: \mathbb{R} \to \mathbb{R}$ differenzierbar und streng monoton wachsend. $\implies$ $f'(x) > 0$ $(x \in \mathbb{R})$.\\
Falsch.\\
Gegenbeispiel: $f(x) = x^3$

Seien $f,g$ differenzierbar.
Ferner sei $f'(x) > 0$ und $g'(x) < 0$ für alle $x \in \mathbb{R}$.
Dann ist $f \circ g$ streng monoton fallend.\\
Wahr.\\
Bew.: Kettenregel

Es sei $f \in C([0,1])$.
Dann ist $f$ differenzierbar in $(0,1)$.\\
Falsch.\\
Begründung: Nur die Rückrichtung gilt.

\subsection{Beziehungen}
Es sei $f: \mathbb{R} \to \mathbb{R}$ eine Funktion.
\begin{displaymath}
  f \text{ ist differenzierbar. } \fbox{\rule{1in}{0pt}\rule[-0.5ex]{0pt}{4ex}} \text{ $f$ ist stetig.}
\end{displaymath}
Lösung: $\Rightarrow$

\begin{displaymath}
  f \text{ ist differenzierbar. } \fbox{\rule{1in}{0pt}\rule[-0.5ex]{0pt}{4ex}} \text{ $f$ ist Lipschitz-stetig.}
\end{displaymath}
Lösung: keine Beziehung

\section{Das Riemann-Integral}

\subsection{Ableitungen}
Berechnen Sie die Ableitung von
\begin{displaymath}
  g(x) = \int_{0}^{x^2} \left(\int_{0}^{t} \cos(t)e^s ds\right)dt.
\end{displaymath}
Idee: $g(x) = F(x^2) - F(0)$\\
Lösung: $g'(x) = 2x \cos(x^2) \left(e^{x^2} - 1\right)$

\subsection{Integrale ausrechnen}
\begin{displaymath}
  \int_{0}^{1} xe^{x + 1} = e
\end{displaymath}
Idee: $e$ rausziehen und partielle Integration mit $f' = e^x$ und $g = x$

\begin{displaymath}            
  \int_{1}^{\pi + 1} t\sin(t - 1)dt = \pi + 2                                   
\end{displaymath}
Idee: (Index-Shift), partielle Integration mit $f' = \sin(t - 1)$ und $g = t$

\begin{displaymath}
  \int_{1}^{\infty} \frac{\log^2(x)}{x^2} dx = 2
\end{displaymath}
Idee: Zwei mal partiell integieren mit $f' = \frac{1}{x^2}$ und $g = \log^2(x)$ oder substituiere $t := \log(x)$, also auch $x = e^t$

\begin{displaymath}
  \int_{0}^{1} x^{-\frac{1}{2}} dx = 2
\end{displaymath}
Idee: Grenzwert $\lim_{n \to 0+}$ und normal lösen

\begin{displaymath}
  \int_{0}^{1} x^3e^{x^2} dx = \frac{1}{2}
\end{displaymath}
Idee: Partielle Integration mit $f' = e^{x^2}2x$ und $g = \frac{1}{2}x^2$ oder substituiere $t := x^2$

\begin{displaymath}
  \int_{\frac{\pi}{2}}^{\pi} x \cdot \sin(x) dx = \pi - 1
\end{displaymath}
Idee: Partielle Integration mit $f' = \sin(x)$ und $g = x$

\begin{displaymath}
  \int_{-e}^{-1} \frac{1}{x} dx = -1
\end{displaymath}
Idee: Punktsymmetrie von $\frac{1}{x}$ bzw. Stammfunktion des $\log(\cdot)$ benutzen

\begin{displaymath}
  \int_{0}^{3\pi} x \sin(x) dx = 3\pi
\end{displaymath}
Idee: Partielle Integration mit $f' = \sin(x)$ und $g = x$

\begin{displaymath}
  \int_{e^4}^{\infty} \frac{1}{x[\log\sqrt{x}]^4}dx = \frac{1}{12}
\end{displaymath}
Idee: Substituiere $t := \log\sqrt{x}$ mit $dt = \frac{1}{2x}dx$ und bilde den Grenzwert

\begin{displaymath}
  \int_{1}^{\infty} \cos\left(\frac{1}{x}\right) \frac{1}{x^2} dx = \sin(1)
\end{displaymath}
Idee: Substituiere $t := \frac{1}{x}$ mit $dt = -\frac{1}{x^2}dx$

\begin{displaymath}
  \int_{0}^{\pi} \sin^2(x) e^x dx = \frac{2}{5} \left(e^{\pi} - 1\right)
\end{displaymath}
Idee: Zweifache partielle Integration mit $f' = e^x$, $g = \sin^2(x)$ und $f' = e^x$, $g = \sin(x)\cos(x)$ mit dem Wissen von $\cos^2(x) = 1 - \sin^2(x)$

\begin{displaymath}
  \int_{1}^{\infty} \frac{1}{x\sqrt{x - 1}} dx = \pi
\end{displaymath}
Idee: Substituiere $t := \sqrt{x - 1}$, also $x = t^2 + 1$ mit $dt = \frac{1}{2\sqrt{x - 1}}dx$ und $\left(\arctan(x)\right)' = \frac{1}{1 + x^2}$ und $\lim_{n \to \infty} \arctan(n) = \frac{\pi}{2}$

\begin{displaymath}
  \int_{-1}^{2} x^2 e^{x^3 + 1} dx = \frac{1}{3} \left(e^9 - 1\right)
\end{displaymath}
Idee: Substituiere $t := x^3$.

\begin{displaymath}
  \int_{1}^{e} x \log(x) dx = \frac{1}{4} \left(e^2 + 1\right)
\end{displaymath}
Idee: Partielle Integration mit $f' = x$ und $g = \log(x)$

\begin{displaymath}
  \int_{0}^{1} t \cdot \arctan(t) dt = \frac{\pi}{4} - \frac{1}{2}
\end{displaymath}
Idee: Partielle Integration mit $f' = t$ und $g = \arctan$ mit $\left(\frac{t^2 + 1}{2}\right)' = f$

\begin{displaymath}
  \int_{0}^{\infty} xe^{-tx} dx = \frac{1}{t^2}
\end{displaymath}
Idee: Partielle Substitution mit $f' = e^{-tx}$ und $g = x$. Grenzwertbetrachtung mit l'Hospital

\subsection{Wahr oder Falsch?}
Die Funktion
\begin{displaymath}
  f: (0,\infty) \to \mathbb{R}, f(x) = \int_{0}^{x} e^{-t^2}dt
\end{displaymath}
ist streng monoton wachsend.\\
Wahr.\\
Begründung: $e^{-t^2} > 0$

Sei $f$ eine nicht Riemann-integrierbare Funktion auf $[a,b]$, $a,b \in \mathbb{R}$, $a < b$.
Dann ist $f$ nicht stetig.\\
Wahr.\\
Begründung: Jede stetige Funktion ist auch integrierbar.

Sei $f:[a,b] \to \mathbb{R}$ eine Riemann-integrierbar.
Dann besitzt $f$ auch eine Stammfunktion.\\
Falsch.\\
Bemerkung im Skript.

Das uneigentliche Riemann-Integral $\int_{0}^{\infty} f(x)dx$ existiere.
Dann gilt $\lim\limits_{x \to \infty} f(x) = 0$.\\
Falsch.\\

Jede differenzierbare Funktion $f:[0,1] \to \mathbb{R}$ ist Riemann-integrierbar.\\
Wahr.\\
Beweis: Jede differenzierbare Funktion ist stetig und jede stetige Funktion ist riemann-integrierbar.

Das Produkt zweier Riemann-integrierbaren Funktionen ist wieder Riemann-integrierbar.\\
Wahr.\\
Beweis: Satz aus der Vorlesung

Sei $f: \mathbb{R} \to \mathbb{R}$ stetig und $a \in \mathbb{R}$.
Dann gilt
\begin{displaymath}
  \lim_{\varepsilon \to 0} \int_{a}^{a + \varepsilon} f(x)dx = 0.
\end{displaymath}
Wahr.\\

Sei $f : [0, \infty)\to [0,\infty)$ stetig und es gelte $\int_{0}^{x} f(t)dt \leq 100$ für alle $x \in [0, \infty)$.
Dann ist $f$ uneigentlich Riemann-integrierbar auf $[0,\infty)$.
Wahr.\\

Sei $f_n$ eine Folge in $R([0,1])$ und $f_n$ konvergiert für $n \to \infty$ auf $[0,1]$ punktweise gegen $f: [0,1] \to \mathbb{R}$.
Dann gilt
\begin{displaymath}
  \lim_{n \to \infty} \int_{0}^{1} f_n(x)dx = \int_{0}^{1} f(x)dx.
\end{displaymath}
Falsch.\\
Begründung: Das gilt, falls $f_n$ gleichmäßig konvergiert (und nicht nur punktweise).

Ist $f \in R([0,1])$, so besitzt $f$ eine Stammfunktion.\\
Falsch.\\
Bemerkung in der Vorlesung.

Es sei $f \in C([0,1])$.
Dann ist $f \in R([0,1])$.\\
Wahr.\\
Begründung: Jede stetige Funktion ist Riemann-integrierbar.

Sind $f,g \in R([1,a])$ für alle $a > 1$ und gilt $0 \leq f(x) \leq g(x)$ für alle $x \in [1,\infty)$, dann gilt:
\begin{displaymath}
  \int_{1}^{\infty} f(x) dx \text{ divergent. } \implies \int_{1}^{\infty} g(x) dx \text{ divergent. }
\end{displaymath}
Wahr.

Sind $f,g \in R([a,b])$ für $a,b \in \mathbb{R}$, $a < b$, und ist $f(x) \leq g(x)$ für alle $x \in [a,b]$, dann gilt:
\begin{displaymath}
  \int_{a}^{b} f(x) dx \leq \int_{a}^{b} g(x) dx.
\end{displaymath}
Wahr.

Seien $a,b \in \mathbb{R}$, $a < b$ und $f \in C([a,b])$ und $f(x) \geq 0$.
Dann gilt
\begin{displaymath}
  \int_{a}^{b} f(t) dt = 0 \implies f(x) = 0 \text{ für alle } x \in [a,b].
\end{displaymath}
Wahr.

\subsection{Beziehungen}
Es sei $f: \mathbb{R} \to \mathbb{R}$ eine Funktion.
\begin{displaymath}
  f \text{ ist stetig. } \fbox{\rule{1in}{0pt}\rule[-0.5ex]{0pt}{4ex}} \text{ $f$ besetzt eine Stammfunktion.}
\end{displaymath}
Lösung: $\Rightarrow$ (Hauptsatz der Differential- und Integralrechnung)

\section{Die komplexe Exponentialfunktion}
Gilt folgende Aussage?
\begin{displaymath}
  e^{i \frac{9\pi}{2}} + i = 0
\end{displaymath}
Lösung: Nein, denn $e^{i \frac{9\pi}{2}} = i$.

Geben Sie alle Wurzeln aus $-12 + 16i$ an, also $\sqrt{-12 + 16i}$.\\
Idee: $z = x+iy$ mit $z^2 = -12 + 16i$\\
Lösung: $\pm (2 + 4i)$

\section{Fourierreihen}
Berechnen Sie die Fourierkoeffizienten der folgenden $2\pi$-periodischen Funktion, die auf $[\pi, \pi]$ gegeben ist durch:
\begin{displaymath}
  f(t) =
  \begin{cases}
    1 - |t|& \text{für } |t| < 1\\
    0& \text{für } 1 \leq |t| \leq \pi
  \end{cases}
\end{displaymath}
Idee: $f$ ist gerade\\
Lösung: $a_0 = \frac{1}{\pi}$, $a_n = \frac{2}{\pi n^2}(-\cos(n) + 1)$, $b_n = 0$

Sei $f : \mathbb{R} \to \mathbb{R}$ eine $2\pi$-periodische Funktion, die auf $[\pi, \pi)$ gegebn ist durch
\begin{displaymath}
  f(x) =
  \begin{cases}
    -x,& x \in [-\pi,0)\\
    0,& x \in [0,\pi).
  \end{cases}
\end{displaymath}
Berechnen Sie die reelle Fourierreihe $s_f$ von $f$.
In welchen Punkten $x \in \mathbb{R}$ gilt $s_f(x) = f(x)$?\\
Idee: Koeffizienten mithilfe der Formel bestimmen\\
Lösung: $a_0 = \frac{\pi}{2}$, $a_n = \frac{1}{n^2 \pi}(-1 + (-1)^n)$, $b_n = \frac{(-1)^n}{n}$.
Es gilt $s_f(x) = f(x) \iff f \text{ ist stetig in } x \iff x \in \{(2k+1)\}$

Bestimmen Sie die Fourierrkoeffizienten der Funktion $f : [-\pi, \pi) \to \mathbb{R}$, $f(x) := \cos\left(\frac{1}{2}x\right)$, welche $2\pi$-periodisch auf ganz $R$ fortgesetzt werden.
Idee: $f$ gerade, doppelte partielle Integration\\
Lösung: $a_k = \frac{1}{\pi} \cdot \frac{4(-1)^k}{1-4k^2}$, $b_k = 0$

\section{Konvergenz im $\mathbb{R}^n$}
\textit{Keine konkreten Aufgaben dazu}

\section{Grenzwerte bei Funktionen, Stetigkeit}
\subsection{Stetigkeit}
\subsubsection{Wahr oder Falsch?}
Die Funktion $g : \mathbb{R}^2 \to \mathbb{R}$ mit
\begin{displaymath}
  g(x,y) =
  \begin{cases}
    \frac{xy}{x^2 + y^2}& (x,y) \neq (0,0)\\
    0& (x,y) = (0,0)
  \end{cases}
\end{displaymath}
ist stetig in $(0,0)$.\\
Falsch.\\
Begründung: $\left(\frac{1}{k}, \frac{1}{k}\right) \to 0$, aber $\left(\frac{1}{k}, \frac{1}{k}\right) = \frac{1}{2} \neq 0 = g(0,0)$

Sei $f: \mathbb{R}^2 \to \mathbb{R}$ derart, dass für alle $y \in \mathbb{R}$ die Funktion $x \mapsto f(x,y)$ stetig ist und dass für alle $x \in \mathbb{R}$ die Funktion $y \mapsto f(x,y)$ stetig ist.
Dann ist $f$ stetig auf $\mathbb{R}^2$.\\
Falsch.

Sei $A \subseteq \mathbb{R}^n$ beschränkt, $f : A \to \mathbb{R}$ sei stetig.
Dann existiert $\min_{x \in A} f(x)$.\\
Falsch.

Die Funktion $f : \mathbb{R}^2 \setminus \{0\} \to \mathbb{R}$, $f(x,y) = \frac{xy}{x^2 + y^2}$ lässt sich stetig auf $\mathbb{R}^2$ fortsetzten.\\
Falsch.\\
Begründung: $\lim_{(x,y) \to (0,0)} f(x,y)$ existiert nicht, da $f\left(\frac{1}{k}, 0\right) = 0 \neq \frac{1}{2} = f\left(\frac{1}{k}, \frac{1}{k}\right)$

Sei $f:D \to \mathbb{R}^n$ stetig und $D$ kompakt.
Dann ist das Bild $f(D)$ kompakt.\\
Wahr.

\subsection{Grenzwerte}
\subsubsection{Definition}
Es sei $n \geq 1$, $D \subseteq \mathbb{R}^n$ und $f : D \to \mathbb{R}$ eine Funktion.
Weiter sei $x_0 \in \mathbb{R}^n$ ein Häufungspunkt von $D$.
Geben Sie die Definition für den Grenzwert $\lim_{x \to x_0} f(x)$.\\
Lösung:
$f(x) \to a$ für $x \to x_0$ genau dann, wenn für jede Folge $(x_n)_{n = 1}^{\infty}$ aus $D \setminus x_0$ mit der Eigenschaft $x_n \to x_0$ $(n \to \infty)$ gilt:
$f(x_n) \to a$ $(n \to \infty)$ im Sinne von Folgen in $\mathbb{R}$.

\subsubsection{Grenzwerte bestimmen}
\begin{displaymath}
  \lim_{(x,y) \to (0,0} \frac{\sin(xy)}{xy} = 1
\end{displaymath}
Idee: $\lim_{t \to 0} \frac{\sin(t)}{t} = 1$ (?)

\begin{displaymath}
  \lim_{(x,y) \to (0,0} \frac{y^3}{x^2 + y^2} = 0
\end{displaymath}

\begin{displaymath}
  \lim_{(x,y) \to (1,1)} \frac{\sqrt{x^3y + 4 - x^2} - 2}{xy - 1} = \frac{1}{4}
\end{displaymath}
Idee: Setzte $x = 1$ und löse mit l'Hospital

\begin{displaymath}
  \lim_{(x,y) \to (0,2)} \frac{\sin(xy)}{3x} = \frac{2}{3}
\end{displaymath}
Idee: Setzte $y = 2$ ein und löse mit l'Hospital

\begin{displaymath}
  \lim_{(x,y) \to (0,0)} \frac{x^2(y + 2)}{e^{x^2} - 1} = 2
\end{displaymath}
Idee: Setzte $y = 0$ ein und löse mit l'Hospital

\begin{displaymath}
  \lim_{(x,y,z) \to (0,0,0)} \frac{x^2 y^2}{\sqrt{x^2 + y^2 + z^2}} = 0
\end{displaymath}
Idee:
\begin{displaymath}
  0 \leq \frac{x^2 y^2}{\sqrt{x^2 + y^2 + z^2}}
  \begin{cases}
    = 0,& \text{falls } x = 0,\\
    \leq \frac{x^2 y^2}{\sqrt{x^2}},& \text{ falls } x \neq 0
  \end{cases}
  \leq |x|y^2 \xrightarrow{(x,y) \to (0,1)} 0
\end{displaymath}

\begin{displaymath}
  \lim_{(x,y) \to (0,1)} f(x,y) = 0
\end{displaymath}
mit $f : \{(x,y) \in \mathbb{R}^2: x \neq 0 \neq y\} \to \mathbb{R}$, $f(x,y) = x \log |xy|$\\
Idee:
\begin{displaymath}
  0 \leq |f(x,y)| = \frac{1}{|y|} ||xy| \log |xy|| \xrightarrow{(x,y) \to (0,1)} 0
\end{displaymath}
mit $\lim_{t \to 0^+} t\log(t) = 0$

\begin{displaymath}
  \lim_{(x,y,z) \to (0,0,0)} \frac{xyz^2}{\sqrt{x^8 + y^8 + z^8}} \text{ existiert nicht}
\end{displaymath}
Idee: Betrachte die Folge $\left(\frac{1}{n}, \frac{1}{n}, \frac{1}{n}\right)$ und $\left(\frac{1}{n}, \frac{1}{n}, \frac{2}{n}\right)$

\begin{displaymath}
  \lim_{(x,y) \to (0,0)} \frac{\sin((x-1)y)}{y} = -1
\end{displaymath}
Idee: Nutze $\lim_{t \to 0} \frac{\sin t}{t} = 1$

\begin{displaymath}
  \lim_{(x,y) \to (0,0)} \frac{\sin(2xy) + x^2y}{xy} = 2
\end{displaymath}
Idee: $\lim_{t \to 0} \frac{\sin(t)}{t} = 1$

\begin{displaymath}
  \lim_{(x,y) \to (0,0)} \frac{\sin(x)y^2}{x^3 + y^3} \text{ existiert nicht}
\end{displaymath}
Idee: $\frac{\sin(x)y^2}{x^3 + y^3} = \frac{\sin(x)}{x} \cdot \frac{xy^2}{x^3 + y^3}$ und betrachte dann $(x_n, x_n)$, $(0, x_n)$.

\section{Analysis in $\mathbb{C}$}
Bestimmen Sie den Konvergenzradius der Potenzreihe
\begin{displaymath}
  \sum_{n = 0}^{\infty} \left(\frac{1+2i}{4-3i}\right)^n z^{2n}.
\end{displaymath}
Idee: $\limsup_{n \to \infty} \sqrt[n]{|a_n|} = \frac{1}{\sqrt{5}}$ und $y = z^2$ substituieren\\
Lösung: $R = \sqrt[4]{5}$

Bestimmen Sie den Imaginärteil von
\begin{displaymath}
  \sum_{n = 1}^{\infty} \left(\frac{i}{3}\right)^n.
\end{displaymath}
Idee: geom. Reihe\\
Lösung: $\frac{3}{10}$

\begin{displaymath}
  \sum_{n = 5}^{\infty} \left(\frac{3n+i}{n + 2in}\right)^n \text{ existiert nicht.}
\end{displaymath}
Idee: Betrag betimmen

\section{Differentialrechnung im $\mathbb{R}^n$ (reellwertige Funktionen)}
\subsection{Extremum}
Es sei $f : \mathbb{R}^2 \to \mathbb{R}, f(x,y) = -x^2 + 2x + y^2$.
Zeigen Sie, dass $f$ kein lokales Extremum hat.\\
Lösung:
Nehme an, dass $f$ ein Extremum besitzt.
Mit der notwendigen Bedingung folgt, dass es in $(1,0)$ sein muss.
Die Hesse-Matrix ist allerdings indefinit, also gibt es keine Extrema.

Sei $f : \mathbb{R}^2 \to \mathbb{R}$ definiert durch
\begin{displaymath}
  f(x,y) = 3x(1 - y^2) + \frac{3}{2}x^2.
\end{displaymath}
Bestimmen Sie alle lokalen Extremstellen von $f$.
Besitzt $f$ ein globales Minimum?
Begründen Sie.\\
Lösung:
Fallunterscheidung für die kritischen Stellen, Hesse-Matrix auswerten (nur eine ist positiv definit, also Minimum).\\
Es existiert kein globales Minumum, da $f(1,y) \to -\infty \quad (y \to \infty)$.

Sei $D := \{(x,y) \in \mathbb{R}^2 : x^2 + y^2 \leq 4\}$ und $f : D \to \mathbb{R}$ definiert durch
\begin{displaymath}
  f(x,y) := (x - 3)^2 + y^2.
\end{displaymath}
Bestimmen Sie $\min f(D)$ sowie $\max f(D)$ (mit Beweis).\\
Lösung:
$D$ abgeschlossen und beschränkt, also kompakt und $f$ ist stetig auf $D \implies \min f(D)$ und $\max f(D)$ existiert.
Da es in $D$ keine kritischen Stellen gibt, Randbetrachtung über Polarkoordinaten $x = 2 \cos \phi, \quad y = 2 \sin \phi$ mit $g(\phi) := f(x,y)$.
Nullstellen von $g$ betrachten ($0$, $\pi$) und in $g$ einsetzen.
Das entspricht $\min f(D)$ bzw. $\max f(D)$.

Bestimmen Sie alle stationären Punkte der Funktion $f : \mathbb{R}^2 \to \mathbb{R}$,
\begin{displaymath}
  f(x,y) = (y^2 - 1) \sin(x)
\end{displaymath}
Hat $f$ ein globales Maximum oder ein globales Minimum?\\
Lösung:
Mit einer Fallunterscheidung folgen die stationären Punkte
\begin{displaymath}
  \left(\frac{\pi}{2}\left(2k + 1\right), 0\right), (nk, \pm 1) \quad (k \in \mathbb{Z})
\end{displaymath}
$f$ hat weder Minimum noch Maximum, denn
\begin{align*}
  f\left(\frac{\pi}{2}, y\right) &= y^2 - 1 \to \infty &\quad (y \to \infty)\\
  f\left(-\frac{\pi}{2}, y\right) &= -(y^2 - 1) \to -\infty &\quad (y \to \infty)
\end{align*}

Berechnen Sie das globale Minimum und das globale Maximu der Funktion $f : \{(x,y) \in \mathbb{R}^2 : x,y \geq 0, x + y \leq 1\} \to \mathbb{R}$, $f(x,y) = (x-y)e^{x+y}$.\\
Lösung:
Sei $M := \{(x,y) \in \mathbb{R}^2 : x,y \geq 0, x + y \leq 1\}$.
Da $M$ abgeschlossen und beschärnkt also kompakt, $f$ stetig ist, muss $f$ sein Minimum und Maximum auf $M$ annehmen.
$f$ ist in der offenen Menge (ohne Rand) offenbar differenzierbar.
Allerdiengs hat der Gradient keine Nullstellen.
Also muss $f$ Minimum und Maximum auf dem Rand annehmen.
Die Ränder ergeben für $x,y \in [0,1]$
\begin{align*}
  f(0,y) =& -ye^y \in [-e,0],\\
  f(x,0) =& xe^x \in [0,e]\\
  f(x,1-x) =& (2x - 1)e \in [-e,e]
\end{align*}
Aufgrund von $f(0,1) ) -e$ und $f(1,0) = e$ ergibt sich das globale Minimum bzw. globale Maximum von $f$.

Es sei $f : \mathbb{R}^2 \to \mathbb{R}$, $f(x,y) = x^2 + xy + y^2 -3x -3y$.
Zeigen Sie, dass $f$ genau ein lokales Extremum hat, berechnen Sie dieses und geben Sie dessen Art an.\\
Weiter sei $M := \{(x,y) \in \mathbb{R}^2 : x^2 + y^2 < 10\}$.
Werden $\sup f(M)$ und $\inf f(M)$ in $M$ angenommen?
Beweisen Sie Ihre Aussagen.\\
Lösung:
Kritische Stelle mit Gradienten bestimmen $(1,1)$.
Hessematrix ist positiv definit (nur EW größer null).
Damit folgt ein lokales Minimum.\\
Da $1^2 + 1^2 = 2 < 10$ und $f(x,y) \to \infty$ für $(x,y) \to \infty$, wird $\inf f(M)$ angenommen.
Angenommen $\sup f(M)$ wird angenommen.
Dann gibt es eine Umgebung $U \subset M$ (da $M$ offen) mit $f(x,y) \geq f(z,w)$ mit $(z,w) \in U$.
Damit wäre $(x,y)$ ein lokales Maximum mit $f'(x,y) = 0$.
Also ist $(x,y) = (1,1)$ und damit $f$ konstant auf $M$.
Widerspruch.

Sei $A := [0,1] \times [0,1]$ und sei $f : A \to \mathbb{R}$ gegeben durch
\begin{displaymath}
  f(x,y) = e^x - xy - y^2.
\end{displaymath}
Bestimmen Sie (mit Beweis) $\max_{(x,y) \in A} f(x,y)$ und $\min_{(x,y) \in A} f(x,y)$.\\
Lösung:
Da $f$ stetig und $A$ kompakt, wird Minimum und Maximum angenommen.
Es gibt keine lokalen Extrema, also wird Min/Max auf dem Rand angenommen.
$\partial A = ([0,1] \times \{0\}) \cup (\{1\} \times [0,1]) \cup ([0,1] \times \{1\}) \cup (\{0\} \times [0,1])$.
Nach Auswertung des Rands (8 Fälle) ergibt sich $\max f(x,y) = e$ und $\min f(x,y) = 0$.

Untersuchen Sie die Funktion $f : \mathbb{R}^2 \to \mathbb{R}$,
\begin{displaymath}
  f(x,y) = x^2 - 2x + y^3 - 3y \quad ((x,y) \in \mathbb{R}^2),
\end{displaymath}
auf lokale und globale Extrema.\\
Lösung:
$f$ besitzt keine globalen Extrema, denn $f(0,y) \to +\infty$ für $y \to +\infty$ und $f(0,y) \to -\infty$ für $y \to -\infty$.
Für den stationären Punkt $(1,1)$ ist die Hessematrix positiv definit, also liegt ein lokales Minimum vor.
Für $(1,-1)$ ist die Hessemtraix indefinit, also liegt kein Extremum vor.

Sei $f : \mathbb{R}^2 \to \mathbb{R}$, $f(x,y) = 2x^3 - 3xy + 2y^3 - 3$.
Bestimmen Sie
\begin{itemize}
    \item $\grad f(x,y) = (6x^2 - 3y, -3x + 6y^2)$
    \item $\grad f(x,y) = 0 \iff (x,y) \in \{(0,0), \left(\frac{1}{2}, \frac{1}{2}\right)\}$
    \item Die Hessematrix von $f$ ist $H_f(x,y) = \begin{pmatrix} 12x & -3\\-3 & 12y\end{pmatrix}$
    \item $f$ hat in $(x,y)$ ein lokales Minimum $\iff (x,y) = \left(\frac{1}{2}, \frac{1}{2}\right)$
\end{itemize}

Seien
\begin{displaymath}
  K := \{(x,y) \in \mathbb{R}^2 : x\in [0,10], 0 \leq 10 - x\}
\end{displaymath}
und
\begin{displaymath}
  f : K \to \mathbb{R}, \quad f(x,y) := - \frac{1}{x + 1} + y^2 - 2 + 20x.
\end{displaymath}
Bestimmen Sie das globale Maximum und Minimum der Funktion $f$ sowie alle Punkte, in denen diese angenommen werden.\\
Lösung:
$f(0,0) = -3$ lokales Minimum und $f(10,0) = 197 \frac{10}{11}$ globales Maximum
(September 2015)

\subsection{Beziehungen}
Es sei $D \subseteq \mathbb{R}^n$ offen und $f : D \to \mathbb{R}$ differenzierbar auf D.
\begin{displaymath}
  f'(x) = 0 \quad (x \in D). \quad \fbox{\rule{1in}{0pt}\rule[-0.5ex]{0pt}{4ex}} \quad f \text{ ist auf } D \text{ konstant}.
\end{displaymath}
Antwort: $\Leftarrow$

Es sei $f \in C(\mathbb{R}^2, \mathbb{R})$.
\begin{displaymath}
  H_f(x_0) \text{ ist positiv definit.} \quad \fbox{\rule{1in}{0pt}\rule[-0.5ex]{0pt}{4ex}} \quad f \text{ hat in } x_0 \text{ ein lokales Minimum.}
\end{displaymath}
Antwort: keine Beziehung

Es sei $D \subseteq \mathbb{R}^n$ offen und $f : D \to \mathbb{R}$.
\begin{align*}
  \text{Für jedes } x \in D \text{ existiert ein } a \in \mathbb{R}^n \text{, sodass gilt: }&\\
  \frac{|f(x + h) - f(x) - a^T \cdot h|}{||h||} \to 0 \quad (||h|| \to 0)& \quad \fbox{\rule{1in}{0pt}\rule[-0.5ex]{0pt}{4ex}} \quad f \text{ ist stetig partiell differenzierbar.}
\end{align*}
Antwort: $\Leftarrow$

Es sei $f : \mathbb{R}^2 \to \mathbb{R}$.
\begin{displaymath}
  f \text{ ist stetig in } (0,0).
  \quad \fbox{\rule{1in}{0pt}\rule[-0.5ex]{0pt}{4ex}} \quad
  f \text{ istt partiell differenzierbar in } (0,0).
\end{displaymath}
Antwort: keine Beziehung

\subsection{Richtungsableitung}
Es sei $f : \mathbb{R}^2 \to \mathbb{R}, f(x,y) := x^4 y^4$, $a := \frac{1}{\sqrt{10}}(-3,1) \in \mathbb{R}^2$, dann gilt:
\begin{displaymath}
  \frac{\partial f}{\partial a} \left(2,1\right) = f'(2,1) \cdot a = -\frac{32}{\sqrt{10}}.
\end{displaymath}

Sei $f : (0,\infty)^3 \to \mathbb{R}$, $f(x,y,z) := x^y + z$.
Sei $a = \frac{1}{\sqrt{2}}(1, 0, -1)$.
Dann gilt:
\begin{align*}
  \grad f(x,y,z) &= (yx^{y - 1}, x^y \cdot \log x, 1)\\
  \frac{\partial f}{\partial a} (1,1,1) &= 0
\end{align*}

Sei $f : \mathbb{R}^2 \to \mathbb{R}$ gegeben durch $(x,y) \mapsto e^x + y$.
Geben Sie die Richtungsableitung $\frac{\partial f}{\partial a}(0,0)$ für $a = \frac{1}{\sqrt{2}} (1,1)$.
\begin{displaymath}
  \frac{\partial f}{\partial a} (0,0) = \sqrt{2}
\end{displaymath}

Es sei $f : \mathbb{R}^2 \to \mathbb{R}$, $f(x,y) = x^2y^3$, $a := \left(\frac{3}{5}, -\frac{4}{5}\right)$.
\begin{displaymath}
  \frac{\partial f}{\partial a}(x,y) = \frac{6}{5}xy^3 - \frac{12}{5}x^2y^2
\end{displaymath}

Es sei $f : \mathbb{R}^2 \to \mathbb{R}$, $f(x,y) := xy^2$, $a := \left(\frac{1}{\sqrt{2}, \frac{1}{\sqrt{2}}}\right) \in \mathbb{R}^2$, dann gilt
\begin{displaymath}
  \frac{\partial f}{\partial a}(1, -1) = -\frac{1}{\sqrt{2}}
\end{displaymath}

\subsection{Wahr oder Falsch?}
Ist $f : \mathbb{R}^n \to \mathbb{R}$ in $x_0 \in \mathbb{R}^n$ stetig, so ist auch $f$ in $x_0$ stetig partiell differenzierbar.\\
Falsch.

Ist $f : \mathbb{R}^n \to \mathbb{R}$ in $x_0 \in \mathbb{R}^n$ differenzierbar und hat $f$ in $x_0$ ein lokales Maximum, so gilt $\grad f(x_0) = 0$.\\
Wahr.

Seien $f : \mathbb{R}^n \to \mathbb{R}$ und $v_1, \dots, v_n$ linear unabhängige Richtungsvektoren.
$\frac{\partial f}{\partial v_i}(x)$ existiere für  alle $x$ in einer Umgebung von $x_0 \in \mathbb{R}^n$ und sei dort stetig.
Dann ist $f$ in $x_0$ differenzierbar.\\
Wahr.

Sei $f : \mathbb{R}^n \to \mathbb{R}$ zweimal partiell differenzierbar.
Dann ist die Hessematrix $H_f(x)$ in jedem $x \in \mathbb{R}^n$ symmetrisch.\\
Falsch.

Sei $\emptyset \neq D \subseteq \mathbb{R}^n$ offen und konvex und $f : D \to \mathbb{R}$ differenzierbar.
Es existiere ein $C > 0$ mit $|| f'(x) || \leq C$ für alle $x \in D$.
Dann ist $f$ Lipschitz-stetig auf $D$.\\
Wahr.

Sei $f$ für die folgenden 4 Aufgaben $f : \mathbb{R}^2 \to \mathbb{R}$.

Falls $f$ zweimal stetig differenzierbar ist, so gilt
\begin{displaymath}
  \frac{\partial^2 f}{\partial x \partial y}(x,y) - \frac{\partial^2 f}{\partial y \partial x}(x,y) = 0 \quad ((x,y) \in \mathbb{R}^2).
\end{displaymath}
Wahr.

Ist $f$ in $(0,0)$ differenzierbar, so gilt
\begin{displaymath}
  \frac{f(h_1, h_2) - f(0,0) - (h_1,h_2) \cdot \grad f(0,0)}{||(h_1, h_2)||} \to 0, \quad \text{falls } ||(h_1, h_2)|| \to 0.
\end{displaymath}
Wahr.

Gilt
\begin{displaymath}
  \lim_{(h_1, h_2) \to (0,0)} \frac{f(h_1, h_2) - f(0,0) - (h_1,h_2) \cdot \grad f(0,0)}{||(h_1, h_2)||} = 0,
\end{displaymath}
so ist $f$ in $(0,0)$ differenzierbar.\\
Wahr.

Ist für jedes festes $x\in \mathbb{R}$ die Funktion $y \mapsto f(x,y)$ stetig differenzierbar und für jedes festes $y \in \mathbb{R}$ die Funktion $x \mapsto f(x,y)$ stetig differenzierbar, so ist $f$ differenzierbar auf $\mathbb{R}^2$.\\
Wahr.

Gilt für die Hessenatrix von $f$
\begin{displaymath}
  H_f(x) = \begin{pmatrix} 0 & 0 \\ 0 & 0\end{pmatrix},
\end{displaymath}
so hat $f$ in $(0,0)$ kein Maximum.\\
Falsch.

Für $f(x,y) = \frac{x}{1 +y^2} \cos(y)$ gilt $f_{xy}(x,y) = f_{yx}(x,y) \quad ((x,y) \in \mathbb{R}^2)$.\\
Wahr. (Satz von Schwarz)

Sei $f : \mathbb{R}^n \to \mathbb{R}$ in $x_0$ differenzierbar. $\quad \implies \quad f$ ist in $x_0$ stetig.\\
Wahr.

Ist $f : \mathbb{R}^n \to \mathbb{R} \text{ } (n \in \mathbb{N})$ differenzierbar, so ist $f$ auch stetig.\\
Wahr, bekannte Aussage aus der Vorlesung.

Ist $f : \mathbb{R}^n \to \mathbb{R} \text{ } (n \in \mathbb{N}$ mit $n \geq 2)$ partiell differenzierbar, so ist $f$ auch differenzierbar.\\
Falsch.
Denn die Funktion
\begin{displaymath}
  f : \mathbb{R}^2 \to \mathbb{R}, \quad f(x,y) =
  \begin{cases}
    \frac{xy^2}{x^2+y^4},& \text{ falls } (x,y) \neq (0,0)\\
    0,& \text{ falls } (x,y) = (0,0)
  \end{cases}
\end{displaymath}
ist überall partiell differenzierbar, aber in $(0,0)$ nicht stetig.

Ist $f : \mathbb{R}^2 \to \mathbb{R}$ eine zweimal partiell differenzierbare Funktion, so gilt
\begin{displaymath}
  \frac{\partial^2 f}{\partial x \partial y} = \frac{\partial^2 f}{\partial y \partial x}.
\end{displaymath}
Falsch.
Denn die Funktion
\begin{displaymath}
  f : \mathbb{R}^2 \to \mathbb{R}, \quad (x,y) \mapsto
  \begin{cases}
    \frac{xy(x^2 - y^2)}{x^2 + y^2},& \text{ falls } (x,y) \neq (0,0),\\
    0,& \text{ falls } (x,y) = (0,0),
  \end{cases}
\end{displaymath}
ist in allen Punkten zweimal partiell differenzierbar, aber $\frac{\partial^2 f}{\partial x \partial y}(0,0) \neq \frac{\partial^2 f}{\partial y \partial x}(0,0)$.

Sei $f \in C^1(\mathbb{R}^n, \mathbb{R})$ und $a \in \mathbb{R}^n$, $||a|| = 1$.
Dann berechnet sich die Richtungsableitung $\frac{\partial f}{\partial a}$ gemäß der Formel $\frac{\partial f}{\partial a} = f'(x) \cdot a$.\\
Wahr.

Ist $f \in C^2(\mathbb{R}^n, \mathbb{R})$ und besitzt $f$ in $x_0$ ein lokales Minimum, so ist die Hesse-Matrix $H_f(x)$ positiv definit.\\
Falsch.

Sei $f : \mathbb{R}^2 \to \mathbb{R}$, $f(x,y) = \frac{xy^2}{x^2 + y^4}$, falls $(x,y) \neq 0$ und $f(0,0) = 0$.
Dann gilt für $a = \frac{1}{\sqrt{2}}(1,1)$:
\begin{displaymath}
  \grad f(0,0) \cdot a = \frac{\partial f}{\partial a}(0,0).
\end{displaymath}
Falsch.

Eine Funktion $f : \mathbb{R}^n \to \mathbb{R}$ ist in $x_0$ differenzierbar genau dann, wenn es ein $a \in \mathbb{R}^n$ gibt, so dass der Grenzwert
\begin{displaymath}
  \lim_{h \to 0} \frac{f(x_0 + h) - f(x_0) - a \cdot h}{||h||}
\end{displaymath}
exisitert.\\
Falsch.

Sei $f : \mathbb{R}^n \to \mathbb{R}$ $(n \in \mathbb{N})$ zweimal stetig differenzierbar.
Ist die Hesse-Matrix von $f$ im Punkt $x \in \mathbb{R}^n$ nicht invertierbar, so liegt in $x$ weder ein Maximum noch ein Minimum von $f$ vor.\\
Falsch.

Sei $f : \mathbb{R}^n \to \mathbb{R}$ $(n \in \mathbb{N})$ partiell differenzierbar und alle partiellen Ableitungen sind stetig.
Dann ist $f$ differenzierbar.\\
Wahr.

\subsection{Ableitungen}
Existiert die Ableitung $f'(0,0)$ für
\begin{displaymath}
  f : \mathbb{R}^2 \to \mathbb{R} \text{ mit } f(x,y) = ||(x,y)|| \quad ((x,y) \in \mathbb{R}^2)
\end{displaymath}
Nein, denn aus der Ableitung der ersten Variablen folgt
\begin{displaymath}
  \frac{1}{h} \cdot (f(h, 0) - f(0,0)) = \frac{|h|}{h}
\end{displaymath}
und der Grenzwert $\lim_{h \to 0} \frac{|h|}{h}$ ist nicht eindeutig.

Existiert die Ableitung $f'(0,0$ für
\begin{displaymath}
  f : \mathbb{R}^2 \to \mathbb{R}, \quad f(x,y) :=
  \begin{cases}
    (x^2 + y^2) \sin\left(\frac{1}{\sqrt{x^2 + y^2}}\right),& \text{ falls } (x,y) \neq (0,0),\\
    0,& \text{ falls } (x,y) = (0,0).
  \end{cases}
\end{displaymath}
Für die partielle Ableitung im Punkt $(0,0)$ nach $x$ ergibt sich
\begin{displaymath}
  f_x(0,0) = \lim_{t \to 0} \frac{f(t,0) - f(0,0)}{t} = \lim_{t \to 0} \frac{t^2 \sin(|t|^{-1)} - 0}{t} = \lim_{t \to 0} t \sin(|t|^{-1}) = 0.
\end{displaymath}
Aus Symmetriegründen gilt $f_y(0,0) = f_x(0,0)$, also $\grad f(0,0) = (0,0)$.
Wegen $f(0,0) = 0$ und $\grad f(0,0) \cdot (h_1, h_2) = 0$ für $(h_1, h_2) \in \mathbb{R}^2$ gilt
\begin{displaymath}
  \frac{f(h_1, h_2) - f(0,0) - \grad f(0,0) \cdot (h_1, h_2)}{||(h_1, h_2)||} = ||(h_1, h_2)|| \sin \frac{1}{||(h_1, h_2)||} \to 0 \quad (h_1, h_2) \to 0.
\end{displaymath}
Somit ist f auch differenzierbar in $(0,0)$ mit $f'(0,0) = (0,0)$.

\section{Differentialrechnung im $\mathbb{R}^n$ (vektorwertige Funktionen)}
\subsection{Ableitungen}
Bestimme $f'(x,y)$ für $f : \mathbb{R}^2 \to \mathbb{R}^3$ mit $f(x,y) = (2 + x \cos y, e^{x^2 - y^2}, y^3 x)$.
\begin{displaymath}
  f'(x,y) = 
  \begin{pmatrix}
    \cos y & -x \sin y\\
    2x e^{x^2 - y^2} & -2y e^{x^2 - y^2}\\
    y^3 & 3xy^2
  \end{pmatrix}
\end{displaymath}

Bestimme $f'(x,y)$ für $f : \mathbb{R}^2 \to \mathbb{R}^3$ mit $f(x,y) = (x + y, xy\cos(y^2), y^3)$.
\begin{displaymath}
  f'(x,y) = 
  \begin{pmatrix}
    1 & 1\\
    y\cos(y^2) & x(\cos(y^2) - 2y^2\sin(y^2))\\
    0 & 3y^2
  \end{pmatrix}
\end{displaymath}

Berechnen Sie die Jacobi-Matrix der Funktion $f(x,y) = (x^3y, 1 - \cos(y), \sin(x)e^y)$.
\begin{displaymath}
  f'(x,y) =
  \begin{pmatrix}
    3x^2y & x^3\\
    0 & \sin(y)\\
    \cos(x)e^y & \sin(x)e^y
  \end{pmatrix}
\end{displaymath}

Berechen Sie die Ableitung (falls sie existiert) für $g : \mathbb{R}^3 \to \mathbb{R}^2$ mit $g(x,y,z) = (z \cos(xy^2), e^x + y \log(z^2 + 1))$.
\begin{displaymath}
  g'(x,y,z) = 
  \begin{pmatrix}
    -y^2z \sin(xy^2) & -2xyz \sin(xy^2) & \cos(xy^2)\\
    e^x & \log(z^2 + 1) & \frac{2zy}{z^2 + 1}
  \end{pmatrix}
\end{displaymath}

Berechnen Sie die Jacobi-Matrix der Funktion
\begin{displaymath}
  f(x,y,z) = (x^2 + \sin(y) + z, \cos(x) - y + z^2)
\end{displaymath}
\begin{displaymath}
  f'(x,y,z) =
  \begin{pmatrix}
    2x & \cos(y) & 1\\
    -\sin(x) & -1 & 2z
  \end{pmatrix}
\end{displaymath}

Es sei $f : \mathbb{R}^3 \to \mathbb{R}^2$, $f(x,y,z) := (e^{xy}z, \sin(z) + x + y)$.
\begin{displaymath}
  f'(x,y,z) =
  \begin{pmatrix}
    e^{xy}yz & e^{xy}xz & e^{xy}\\
    1 & 1 & \cos(z)
  \end{pmatrix}
\end{displaymath}

\subsection{Implizit definierte Funktionen}
Sei $f : \mathbb{R}^3 \to \mathbb{R}$ definiert durch
\begin{displaymath}
  f(x,y,z) = z^3 + 2z^2 - 3xyz + x^3 - y^3.
\end{displaymath}
\begin{enumerate}
    \item Zeigen Sie, dass eine Umgebung $U \subseteq \mathbb{R}^2$ des Punktes $(0,0)$ und eine Umgebung $V \subseteq \mathbb{R}$ des Punktes $-2$ exisitert, sowie genau eine $C^1$-Funktion $g : U \to V$ mit $g(0,0) = -2$ und $f(x,y,g(x,y)) = 0$ für alle $(x,y) \in U$.
    \item Berechnen Sie $g'(0,0)$.
    \item Zeigen Sie, dass $f$ surjektiv ist.
\end{enumerate}
Lösung:
\begin{enumerate}
    \item $f$ ist offenbar stetig partiell differenzierbar auf $\mathbb{R}^3$.
    Weiter gilt:
    \begin{align*}
      f(0,0,-2) = 0 &\quad \checkmark\\
      f_z(0,0,-2) = \frac{\partial f}{\partial z} (0,0,-2) = 4 \neq 0 &\quad \checkmark.
    \end{align*}
    Damit ist der Satz über implizit definierte Funktionen anwendbar.
    \item Für $x,y \in U$ gilt:
    \begin{displaymath}
      g'(x,y) = -\left(\frac{\partial f}{\partial z} (x,y,g(x,y))\right)^{-1} \cdot \frac{\partial f}{\partial (x,y)} (x,y,g(x,y))
    \end{displaymath}
    Daraus folgt: $g'(0,0) = (0,0)$
    \item Sei $h(z) := f(0,0,z) = z^3 + 2z^2$. Mit $\lim_{z \to \infty} h(z) = \infty$ und $\lim_{z \to -\infty} h(z) = -\infty$ und der Stetigkeit von $h : \mathbb{R} \to \mathbb{R}$ folgt mit dem Zwischenwertsatz $h(\mathbb{R}) = \mathbb{R}$.
    Und daher folgt:
    \begin{displaymath}
      \mathbb{R} = h(\mathbb{R}) \subseteq f(\mathbb{R}^3) \subseteq \mathbb{R} \implies f(\mathbb{R}^3) = \mathbb{R} \implies f \text{ surjektiv}
    \end{displaymath}
\end{enumerate}

Beweisen Sie:
Es existiert eine Umgebung $U$ des Punktes $(0,0)$, so dass das Gleichungssystem
\begin{displaymath}
  \begin{cases}
    e^{2x + y} - \cos(xy) &= s\\
    e^x - \cos(x + y) &= t
  \end{cases}
\end{displaymath}
für jeden Punkt $(s,t) \in U$ eine eindeutige Lösung $x := x(s,t)$, $y := y(s,t)$ mit $x(0,0) = 0$ und $y(0,0) = 0$ besitzt.\\
\textit{Hinweis}: Betrachten Sie die Funktion $f : \mathbb{R}^4 \to \mathbb{R}^2$,
\begin{displaymath}
  f(s,t,x,y) = (e^{2x + y} - \cos(xy) - s, e^{x} - \cos(x + y) - t).
\end{displaymath}
Lösung:
$f$ ist offensichtlich stetig partiell differenzierbar auf $\mathbb{R}^4$. (\checkmark)
Es gilt $f(0,0,0,0) = (0,0)$. (\checkmark)
\begin{displaymath}
  \frac{\partial f}{\partial (x,y)} (s,t,x,y) =
  \begin{pmatrix}
    2e^{2x + y} + \sin(xy)y & e^{2x+y} + \sin(xy)x\\
    e^x + \sin(x + y) & \sin(x + y)
  \end{pmatrix}
\end{displaymath}
Also
\begin{displaymath}
  \frac{\partial f}{\partial (x,y)} (0,0,0,0) =
  \begin{pmatrix}
    2 & 1\\
    1 & 0
  \end{pmatrix}
  \implies \det = -1 \neq 0 \implies \text{invertierbar} (\checkmark)
\end{displaymath}
Nach dem Satz über implizit definierte Funktionen existieren Umgebungen $U$ von $(0,0)$ und $V$ von $(0,0)$ sowie eine eindeutig bestimmte $C^1$-Funktion $g : U \to V$ mit $g(0,0) = (0,0)$ und $f(s,t,g(s,t)) = (0,0)$ für alle $(s,t) \in U$.
Setzt man $g(s,t) = (x(s,t), y(s,t))$ so gilt $f(s,t,x(s,t), y(s,t)) = (0,0)$ genau dann, wenn $(x(s,t), y(s,t))$ für gegebenes $(s,t) \in U$ das Gleichungssystem löst.

Es sei $y : \mathbb{R} \to \mathbb{R}$ definiert durch $x = y(x) - \frac{1}{2} \sin(y(x))$.
Berechnen Sie $y'(0)$.\\
Lösung: 
$y(0)$ löst die Gleichung $0 = y(0) - \frac{1}{2}\sin(y(0))$, also $y(0) = 0$.
Mit dem Satz über implizit definierte Funktionen folgt daher:
\begin{displaymath}
  y'(0) = \frac{1}{(y(0))'} = \frac{1}{1 - \frac{1}{2} \cos(y(0))} = 2.
\end{displaymath}

\subsection{Umkehrsatz}
Sei $g : \mathbb{R}^2 \to \mathbb{R}^2$ gegeben durch $g(x,y) = (x^2 - y^3, x^4 + y)$.
Bestimmen Sie die Menge $A$ aller Punkte $(x_0, y_0) \in \mathbb{R}^2$, so dass $g$ in einer Umgebung von $(x_0, y_0)$ invertierbar ist.\\
Lösung:
\begin{displaymath}
  A = \{(x_0, y_0) \in \mathbb{R}^2 : x_0 \neq 0\}
\end{displaymath}

Die Funktion $f : (0, \infty) \times \mathbb{R} \to \mathbb{R}^2$ sei gegeben durch
\begin{displaymath}
  f(r, \varphi) := r^2 (\cos(\varphi), \sin(\varphi)).
\end{displaymath}
\begin{enumerate}
    \item Zeigen Sie:
    Es gibt eine Umgebung $U$ von $(2,0)$ und eine Umgebung $V$ von $(4,0)$ so, dass $U$ durch die Funktion $f$ bijektiv auf $V$ abgebildet wird.
    Berechnen Sie die Ableitung der Umkehrfunktion $(f|_U)^{-1}$ in $(4,0)$.
    \item Ist $f$ injektiv?
    Begründen Sie ihre Antwort.
\end{enumerate}
Lösung:
\begin{enumerate}
    \item Der Umkehrsatz liefert die Behauptung, wenn folgende Bedingungen erfüllt werden:
    \begin{itemize}
        \item $f$ ist stetig differenzierbar.
        \item $f(2,0) = (4,0)$
        \item $f'(2,0)$ ist invertierbar
    \end{itemize}
    Weiter gilt
    \begin{displaymath}
      (f^{-1})'(4,0) = (f'(2,0))^{-1} = 
      \begin{pmatrix}
        \frac{1}{4} & 0\\
        0 & \frac{1}{4}
      \end{pmatrix}
    \end{displaymath}
    \item $f$ ist offensichtlich nicht injektiv, denn $f(r, \varphi) = f(r, \varphi + 2\pi)$
\end{enumerate}

Sei $f : \mathbb{R}^2 \to \mathbb{R}^2$ definiert durch
\begin{displaymath}
  f(x,y) = (e^{x + y}, y).
\end{displaymath}
\begin{enumerate}
    \item Zeigen Sie:
    Es gibt eine offene Umgebung $U$ von $(0,1)$ und eine offene Umgebung $V$ von $(e,1)$ so, dass $U$ durch die Funktion $f$ bijektiv auf $V$ abgebildet wird.
    Berechnen Sie die Ableitung der Umkehrfunktion $(f|_U)^{-1} : V \to U$ im Punkt $(e,1)$.
    \item Zeigen Sie, dass $f$ injektiv, aber nicht surjektiv ist.
\end{enumerate}
Lösung: 
\begin{enumerate}
    \item
    \begin{itemize}
        \item $f$ ist offensichtlicheine $C^1$-Funktion.
        \item $f(0,1) = (e,1) \quad \checkmark$
        \item $f'(0,1) = \begin{pmatrix} e & e\\ 0 & 1\end{pmatrix}$, also $\det f'(0,1) = e \neq 0 \quad \checkmark$
    \end{itemize}
    Also ist $f'(0,1)$ invertierbar und der Umkehrsatz findet Anwendung.
    Weiter gilt:
    \begin{displaymath}
      (f^{-1})'(e, 1) = (f'(0,1))^{-1} =
      \begin{pmatrix}
        e^{-1} & -1\\
        0 & 1
     \end{pmatrix}
    \end{displaymath}
    \item Injektivität folgt aus der Injektivität der $e$-Funktion.
    $f$ ist nicht surjektiv, da $(0,0)$ wegen $e^x > 0$ nicht getroffen wird.
\end{enumerate}

Sei $f : \mathbb{R}^3 \to \mathbb{R}^3$ definiert durch
\begin{displaymath}
  f(x,y,z) = \left(e^{x(y + z)}, e^{y(x + z)}, \frac{1}{2}(y^2 + 1)\right).
\end{displaymath}
Zeigen Sie, dass $f$ in einer Umgebung von $(0,1,0)$ invertierbar ist und berechnen Sie $(f^{-1})'(1,1,1)$.
Ist $f : \mathbb{R}^3 \to \mathbb{R}^3$ injektiv?\\
Lösung:
\begin{itemize}
    \item $f$ ist offensichtlich eine $C^1$-Funktion auf $\mathbb{R}^3. \quad \checkmark$
    \item $f(0,1,0) = (1,1,1) \quad \checkmark$
    \item $f'(0,1,0)$ ist invertierbar, denn
    \begin{displaymath}
      f'(0,1,0) = 
      \begin{pmatrix}
        1 & 0 & 0\\
        1 & 0 & 1\\
        0 & 1 & 0
      \end{pmatrix}
      \implies \det f'(0,1,0) = -1 \neq 0. \quad \checkmark
    \end{displaymath}
\end{itemize}
Nach dem Umkehrsatz existieren Umgebungen $U,V \subset \mathbb{R}^3$ von $(0,1,0)$ bzw. $(1,1,1)$, so dass $f : U \to V$ bijektiv und somit invertierbar ist.
Nach dem Umkehrsatz gilt:
\begin{displaymath}
  (f^{-1})'(1,1,1) = (f'(0,1,0))^{-1} =
  \begin{pmatrix}
    1 & 0 & 0\\
    0 & 0 & 1\\
    -1 & 1 & 0
  \end{pmatrix}.
\end{displaymath}
$f$ ist nicht injektiv, da $f(0,1,0) = (1,1,1) = f(0,-1,0)$.

\subsection{Wahr oder Falsch?}
Sei $f : \mathbb{R}^2 \to \mathbb{R}^2$, $f(x,y) := (x^2y, (x-3)y^2)$.
Dann existiert eine offene Umgebung $U$ von $(1,1)$ auf der $f$ injektiv ist.\\
Wahr.

Die Funktion $F : \mathbb{R}^3 \to \mathbb{R}$ gegeben durch $F(x,y,z) = z^3 + 2xy - 4xz + 2y - 1$.
Es gibt eine offene Umgebung $U$ von $(1,1)$ und eine stetig differenzierbare Funktion $\varphi : U \to \mathbb{R}$ mit $F(x,y,\varphi(x,y)) = 0$ für alle $(x,y) \in U$ und mit $\varphi(1,1) = 1$.\\
Wahr.

Ist $\emptyset \neq D \subseteq \mathbb{R}^n$ offen und ist $f : D \to \mathbb{R}^n$ $(n \in \mathbb{N}$ mit $n \geq 2$) stetig partiell differenzierbar und ist $f'(x)$ für jeden Punkt $x \in \mathbb{R}^n$ invertierbar, so ist $f$ injektiv.\\
Falsch.

Sei $f \in C^1(\mathbb{R}^2, \mathbb{R}^2)$ und $\det(f'(x)) \neq 0$ $(x \in \mathbb{R}^2)$. $\quad \implies \quad$ $f$ ist injektiv auf $\mathbb{R}^2$.\\
Falsch.

Sei $f : (-1,1)^2 \to \mathbb{R}$ gegeb durch $f(x,y) = x^2 + \sin(y)$.
Dann gibt es ein $\xi > 0$ und genau eine Funktion $g : (-\xi, \xi) \to \mathbb{R}$ mit $g(0) = 0$ und
\begin{displaymath}
  x^2 + \sin(g(x)) = 0 \quad (x \in (-\xi, \xi)).
\end{displaymath}
Wahr.

Die Funktion $g : \mathbb{R}^n \setminus \{0\} \to \mathbb{R}^n$, $g(x) := \frac{x}{||x||}$ ist differenzierbar.\\
Wahr.

\section{Integration imm $\mathbb{R}^n$}
\subsection{Kreisfläche}
Es sei $B:= \{(x,y) \in \mathbb{R}^2 : x^2 + y^2 \leq e - 1\}$.
Dann gilt
\begin{displaymath}
  \int_{B} \frac{1}{1 + x^2 + y^2}d(x,y) = \pi
\end{displaymath}
Lösung:
Durch einen Übergang in Polarkoordinaten ergibt sich
\begin{displaymath}
  \int_{B} \frac{1}{1 + x^2 + y^2} d(x,y) = \int_{0}^{2\pi} \int_{0}^{\sqrt{e - 1}} \frac{r}{1 + r^2} drd\varphi = 2\pi \frac{1}{2}[\log(1 + r^2)]_{0}^{\sqrt{e-1}} = \pi \log(e) - 0 = \pi.
\end{displaymath}

Es sei $B = \{(x,y) \in \mathbb{R}^2 : x,y \geq 0, x^2 + y^2 \leq 1\}$.
Berechnen Sie
\begin{displaymath}
  \int_B \frac{2}{x^2 + y^2 + 1} \diff (x,y).
\end{displaymath}
Lösung:
Der Übergang in Polarkoordinaten liefert
\begin{displaymath}
  \int_B \frac{2}{x^2 + y^2 + 1} \diff (x,y) = \int_0^{\frac{\pi}{2}} \int_0^1 \frac{2r}{r^2 + 1} \diff r \diff \varphi = \dots = \frac{\pi \log 2}{2}.
\end{displaymath}

Es se $M := \{(x,y) \in \mathbb{R}^2 : x^2 + y^2 \leq 2, |x| > y\}$.
Berechnen Sie
\begin{displaymath}
  \int_M (x^2 + y^2) \diff (x,y).
\end{displaymath}
Der Übergang in Polarkoordinaten liefert
\begin{displaymath}
  \int_M (x^2 + y^2) \diff (x,y) = \int_{-\frac{5}{4}\pi}^{\frac{1}{4} \pi} \int_0^{\sqrt{2}} r^2 \cdot r \diff r \diff \varphi = \dots = \frac{3}{2} \pi.
\end{displaymath}

Das Volumen der Menge $B := \{(x,y,z) \in \mathbb{R}^3 : 0 \leq z \leq \pi, x^2 + y^2 \leq \sin^2(z)\}$.
\begin{displaymath}
  |B| = \int_0^{\pi} \int_0^{2\pi} \int_0^{\sin(z)} r \diff r \diff \varphi \diff z = \dots \text{ partielle Integration } \dots = \frac{\pi^2}{2}.
\end{displaymath}

\subsection{Satz von Fubini}
Sei $B := \{(x,y,z) \in \mathbb{R}^3 : 0 \leq x \leq 1, x^2 \leq y \leq \sqrt{x}, 0 \leq z \leq 1\}$.
Berechnen Sie das folgende Integral:
\begin{displaymath}
  \int_B (xyz) d(x,y,z)
\end{displaymath}
Lösung:
\begin{displaymath}
  \int_B (xyz) d(x,y,z) = \int_0^1 \int_0^1 \int_{x^2}^{\sqrt{x}} (xyz) \diff y \diff x \diff z = \dots = \frac{1}{24}.
\end{displaymath}

Sei $D = \{(x,y) \in \mathbb{R}^2 : y \in [0,2], y \leq x \leq y^2 + 1\}$.
Berchnen Sie den Flächeninhalt $|D|$.
\begin{displaymath}
  |D| = \int_D 1 \diff (x,y) = \int_0^2 \int_y^{y^2 + 1} 1 \diff x \diff y = \frac{8}{3}.
\end{displaymath}

Sei $C := [0,1] \times [0,2]$.
\begin{displaymath}
  \int_C (y - \sin(x)) \diff(x,y) = \int_0^1 \int_0^2 (y - \sin(x)) \diff y \diff x = \dots = 2 \cos(1)
\end{displaymath}

\begin{displaymath}
  \int_{[0,1]^2} (xy + y^2) \diff(x,y) = \int_0^1 \int_0^1 (xy + y^2) \diff y \diff x = \dots = \frac{7}{12}
\end{displaymath}

\subsection{Normalbereich und Cavalieri}
Es sei $M := \{(x,y) \in \mathbb{R}^2 : x^2 \geq y \geq 2x^2 - 4, y \geq 0\}$.
\begin{displaymath}
  \int_M 1 \diff (x,y) = \int_0^2 \int_{2x^2 - 4}^{x^2} 1 \diff x \diff y = \frac{16}{3}
\end{displaymath}

Es sei $\triangle \subseteq \mathbb{R}^2$ das Dreieck mit den Ecken $(0,0), (1,0)$ und $(0,1)$.
Berechnen Sie
\begin{displaymath}
  \int_{\triangle} x^2e^y \diff (x,y) = \int_0^1 \int_0^{1 - x} x^2 e^y \diff y \diff x = \dots = 2e - \frac{16}{3}.
\end{displaymath}
Hinweis: partielle Integration

Es sei $B := \{(x,y) \in \mathbb{R}^2 : x^2 - 2x \leq y \leq x\}$.
Berechnen Sie $\vol(B)$.\\
Lösung:
\begin{displaymath}
  \vol(B) = \int_B 1 \diff(x,y) = \int_0^3 \int_{x^2 - 2x}^{x} 1 \diff y \diff x = \dots = \frac{9}{2}.
\end{displaymath}

Es sei $B = \{(x,y) \in \mathbb{R}^2 : x \geq 0, \frac{1}{4}x^2 - 1 < y < 2 - x\}$
Dann gilt
\begin{displaymath}
  \int_B 1 \diff(x,y) = \int_0^2 \int_{\frac{1}{4}x^2 - 1}^{2 - x} 1 \diff y \diff x = \dots = \frac{10}{3}.
\end{displaymath}

\subsection{Wahr oder Falsch?}
Das Volumen der Einheitskugel im $\mathbb{R}^3$ beträgt $\frac{4\pi}{3}$.\\
Wahr.

Sei $g \in C[0,1]$ und $Q$ der durch
\begin{displaymath}
  Q := \{(x,y,z) \in \mathbb{R}^3 : y^2 + z^2 \leq g^2(x), x \in [0,1]\}
\end{displaymath}
definierte Rotationskörper.
Dann ist das Volumen von $Q$ gegeben durch $|Q| = \pi \int_{0}^{1} g^2(x)dx$.\\
Wahr.

Sei $D = \{(x,y,z) \in \mathbb{R}^3 : z \in [0,1], x^2 + y^2 \leq 1\}$.
Dann gilt mit Zylinderkoordinaten:
\begin{displaymath}
  \int_D (x^2 + y^2)e^x d(x,y,z) = \int_0^1 \int_0^{2\pi} \int_0^1 r^2 e^z \diff r \diff \varphi \diff z
\end{displaymath}
Falsch.

Sei $f : [0,3]^2 \to [-2,2]$ stetig.
Dann gilt:
\begin{displaymath}
  |\int_{[0,3]^2} f(x) \diff x | \leq 18
\end{displaymath}
Wahr.

Sei $f : [0,1] \times [0,1] \to \mathbb{R}$ stetig.
Dann gilt
\begin{displaymath}
  \int_0^1 \int_0^y f(x,y) \diff x \diff y = \int_0^1 \int_0^x f(x,y) \diff y \diff x.
\end{displaymath}
Falsch.

Sei $a < b$, seien $f,g : [a,b] \to \mathbb{R}$ stetig mit $f(x) \leq g(x)$ für alle $x \in [a,b]$ und sei $A \subseteq \mathbb{R}^2$ gegeben durch
\begin{displaymath}
  A := \{(x,y) \in [a,b] \times \mathbb{R} : f(x) \leq y \leq g(x)\}.
\end{displaymath}
Der Flächeninhalt von $A$ berechnet sich gemäß der Formel
\begin{displaymath}
  |A| = \int_a^b g(x) \diff x - \int_a^b f(x) \diff x.
\end{displaymath}
Wahr.

\subsection{Beziehungen}
Es sei $f : [0,1]^2 \to \mathop{R}$ stetig.
\begin{displaymath}
  f(x,y) \geq 0 \quad ((x,y) \in [0,1]^2).
  \quad \fbox{\rule{1in}{0pt}\rule[-0.5ex]{0pt}{4ex}} \quad
  \int_{[0,1]^2} f(x,y) \diff (x,y) \geq 0.
\end{displaymath}
Antwort: $\implies$

\section{Differentialgleichungen 1. Ordnung}
\subsection{Anfangswertprobleme}
Die Lösung des Anfangswertproblems $y'(x) - 2y(x) = 4$, $y(0) = 6$ auf $[0,\infty)$ lautet?\\
Lösung:
\begin{align*}
  y_h &= c e^{2x}\\
  y_p &= -2\\
  \implies y(x) &= ce^{2x} - 2
\end{align*}
Das AWP liefert $c = 8$, also
\begin{displaymath}
  y(x) = 8e^{2x} - 2
\end{displaymath}

Lösen Sie das Anfangswertproblem 
\begin{displaymath}
  y'(x) = 2x \cos(x^2)y(x) + xe^{\sin(x^2)}, \quad y(0) = 1.
\end{displaymath}
Lösung:
\begin{align*}
  y_h &= ce^{\int 2x\cos(x^2) \diff x} \overset{\text{Substitution}}{=} ce^{\sin(x^2)}\\
  y_p &= \frac{1}{2} e^{\sin(x^2)} x^2\\
  \implies y(x) &= e^{\sin(x^2)} \left(c + \frac{x^2}{2}\right)
\end{align*}
Das AWP liefert $c = 1$, also
\begin{displaymath}
  y(x) = e^{\sin(x^2)} \left(1 + \frac{x^2}{2}\right)
\end{displaymath}

Lösen Sie die Differentialgleichung $xy(x)^2 \cdot (xy'(x) + y(x)) = 1$ auf einem geeigneten Interval.\\
\textit{Hinweis:}
Verwenden Sie die Substitution $u(x) = x \cdot y(x)$.\\
Lösung:
Die Gleichung mit $x$ multiplizieren ergibt
\begin{displaymath}
  x^2 y^2 (xy'(x) + y(x)) = x
\end{displaymath}
Mit der Substitution und Trennung der Variablen folgt
\begin{align*}
  u^2(x) \cdot u'(x) &= x\\
  u^2(x) \cdot \frac{\diff u}{\diff x} &= x\\
  \frac{1}{3} u^3(x)& = \frac{3}{2} x^2\\
  u^3(x) &= \frac{3}{2}x^3 + C\\
  u(x) &= \sqrt[3]{\frac{3}{2} x^2 + C}.
\end{align*}
Durch rücksubstituieren folgt die allgemeine Lösung
\begin{displaymath}
  y(x) = \frac{1}{x} \sqrt[3]{\frac{3}{2} x^2 + C}.
\end{displaymath}

Lösen Sie das Anfangswertproblem
\begin{displaymath}
  y'(x) = -2x(1 + x^2)y^3(x), \quad y(0) = 1,
\end{displaymath}
in einer offenen Umgebung von $0$.\\
Lösung:
Mit der Trennung der Variablen folgt
\begin{align*}
  \frac{1}{y^3(x)} \diff y &= -2x(1 + x^2) \diff x\\
  -\frac{1}{2y^2(x)} &= -\frac{1}{2} x^4 + x^2 + C\\
  |y(x)| &= \sqrt{\frac{1}{x^4 - 2x^2 + C}}.
\end{align*}
Mit dem Anfangswert folgt $C = 1$, also ergibt sich die allgemeine Lösung
\begin{displaymath}
  y(x) = \frac{1}{x^2 + 1}.
\end{displaymath}

Lösen Sie das Anfangswertproblem auf $(1, \infty)$:
\begin{displaymath}
  y'(x) = \frac{1}{1 - x} y(x) + x - 1, \quad y(2) = 0.
\end{displaymath}
Lösung:
\begin{align*}
  y_h &= ce^{- \log(x - 1)} = \frac{c}{x - 1}\\
  y_p &= \frac{1}{x - 1} \cdot \int -(x - 1)^2 \diff x = \frac{(x - 1)^2}{3}\\
  \implies y(x) &= \frac{c}{x - 1} + \frac{(x-1)^2}{3}.
\end{align*}
Mit dem Anfangswert folgt $c = -\frac{1}{3}$, also ist die allgemeine Lösung
\begin{displaymath}
  y(x) = -\frac{1}{3}\left(\frac{1}{x - 1} - (x - 1)^2\right).
\end{displaymath}

Lösen Sie das Anfangswertproblem
\begin{displaymath}
  (1 - x)y(x)^2 = x^2 y'(x), \quad y(1) = \frac{1}{2}
\end{displaymath}
auf einem geeigneten Intervall.\\
Lösung: Trennung der Variablen eribt
\begin{displaymath}
  y(x) = \frac{1}{\frac{1}{x} + \log(x) - C}
\end{displaymath}
Der Anfangswert liefert $C = -1$, also folgt damit die allgemeine Lösung
\begin{displaymath}
  y(x) = \frac{1}{\frac{1}{x} + \log(x) + 1} \quad (x \in (0, \infty)).
\end{displaymath}

Geben Sie die Lösung des folgenden Anfangswertproblems an:
\begin{displaymath}
  \begin{cases}
    y'(x) &= xy(x), \quad y : [0,\infty) \to \mathbb{R}\\
    y(0) &= 1.
  \end{cases}
\end{displaymath}
Lösung:
Es gilt
\begin{displaymath}
  y(x) = y_h = e^{\int x \diff x} = e^{\frac{1}{2} x^2}.  
\end{displaymath}

Lösen Sie das folgende Anfangswertproblem und geben Sie ein offenes Interval $I \subseteq \mathbb{R}$ an, auf dem die Lösung definiert ist.
\begin{displaymath}
  y' = -\frac{x}{y}, \quad y(0) = \sqrt{2}.
\end{displaymath}
Lösung:
Mit der Trennung der Variblen folgt
\begin{align*}
  \frac{\diff y}{\diff x} &= -\frac{x}{y}\\
  \frac{1}{2}y^2 &= -\frac{1}{2} x^2 + C\\
  y(x) &= \pm \sqrt{2C - x^2}.
\end{align*}
Mit der Anfangsbedingung folgt $C = 1$ und es ist die positive Wurzel zu nehmen.
Damit folgt die allgemeine Lösung
\begin{displaymath}
  y(x) = \sqrt{2 - x^2} \text{ auf dem Intervall } I = (-\sqrt{2}, \sqrt{2}).
\end{displaymath}

Lösem Sie das folgende Anfangswertproblem auf $[1, \infty)$.
\begin{displaymath}
  y'(x)(1 + x^2) \sinh(y(x)) = 2x \cosh(y(x)). \quad y(1) = 1.
\end{displaymath}
Lösung: Mit der Trennung der Variablen folgt
\begin{displaymath}
  \frac{\sinh(y)}{\cosh(y)} \diff y = \frac{2x}{(1 + x^2)} \diff x.
\end{displaymath}
Beide Seiten lassen sich mithilfe einer Substitution integrieren.
\begin{displaymath}
  \log(\cosh(y)) = \log(1 + x^2) + C
\end{displaymath}
Mit der Injektivität der Logarithmus-Funktion folgt daher
\begin{displaymath}
  \cosh(y) = C(1 + x^2).
\end{displaymath}
Mit dem Anfangswert folgt $C = \frac{1}{2} \cosh(1) = \frac{1}{4}\left(e + \frac{1}{e}\right)$.
Daher lautet die allgemeine Lösung
\begin{displaymath}
  y(x) = \text{arcosh}\left(\frac{1}{4}(e + \frac{1}{e})(1 + x^2)\right).
\end{displaymath}

Lösen Sie das Anfangswertproblem 
\begin{displaymath}
  y'(x) = -\frac{x^3}{y^5}, \quad y(0) = 1
\end{displaymath}
in einer geeigneten Umgebung von 0.\\
Lösung: 
Mit der Trennung der Variablen und anschließender Integration gilt
\begin{displaymath}
  y(x) = \pm \sqrt[6]{C - \frac{3}{2}x^4}.
\end{displaymath}
Mit dem Anfangswert ergibt sich $C = 1$ und die positive Wurzel:
\begin{displaymath}
  y(x) = \sqrt[6]{1 - \frac{3}{2}x^4}.
\end{displaymath}
Jede Umgebung mit $B_r(0)$ mit $r < \sqrt[4]{\frac{2}{3}}$ eignet sich als ``geeignete Umgebung von 0.''

Lösen Sie die folgenden Anfangswertprobleme auf geeigneten Intervallen:
\begin{enumerate}
    \item $y'(x) = \frac{x - 2}{x} y(x) + \frac{2}{x}e^x, \quad y(1) = -e$
    \item $y'(x) = 12x^2(1 - x) \tan(y(x)), \quad y(0) = \frac{\pi}{4}$.
\end{enumerate}
Lösungen:
\begin{enumerate}
    \item 
    \begin{align*}
      y_h &= ce^{x - 2\log(x)} = ce^x x^{-2}\\
      y_p &= \dots = e^x\\
      \implies y(x) &= c\frac{e^x}{x^2} + e^x.
    \end{align*}
    Der Anfangswert liefert $c = -2$, somit ergibt sich die Lösung
\begin{displaymath}
  y(x) = e^x - 2\frac{2}{x^2}e^x.
\end{displaymath}
  \item
  Eine Lösung des Anfangswertproblems erfüllt
  \begin{displaymath}
    \int_0^x \frac{y'(t) \cos(y(t))}{\sin(y(t))} \diff t = \int_0^x 12t^2(1 - t) \diff t = 4x^3 - 3x^4 
  \end{displaymath}
  sowie
  \begin{align*}
    \int_0^x \frac{y'(t) \cos(y(t))}{\sin(y(t))} \diff t &= \int_{\frac{\pi}{4}}^{y(x)} \frac{\cos(s)}{\sin(s)} \diff s\\
    &= [\log(|\sin(s)|)]_{\frac{\pi}{4}}^{y(x)}\\
    &= \log(|\sin(y(x))|) - \log\left(\frac{1}{\sqrt{2}}\right).
  \end{align*}
  Es folgt $|\sin(y(x))| = \frac{1}{\sqrt{2}} e^{4x^3 - 3x^4}$ und damit wegen $0 < y(0) < \pi$
  \begin{displaymath}
    y(x) = \arcsin\left(\frac{1}{\sqrt{2}}e^{4x^3 - 3x^4}\right), \quad (-\varepsilon < x < \varepsilon)
  \end{displaymath}
  für ein $\varepsilon > 0$ hinreichend klein.
\end{enumerate}

\subsection{Wahr oder Falsch?}
Das Anfangswertproblem $y'(x) = Ay(x)$, $y(0) = y_0$ ist für alle $A \in \mathbb{R}^{n \times n}$ und alle $y_0 \in \mathbb{R}^n$ auf $\mathbb{R}$ eindeutig lösbar.\\
Wahr.

Seien $\alpha, s \in C(\mathbb{R})$.
Dann hat die lineare Differentialgleichung $y'(x) = \alpha(x)y(x) + s(x)$, $(x \in \mathbb{R})$ genau eine Lösung.\\
Falsch.

\subsection{Beziehungen}
Es sei $\alpha \in C^1(\mathbb{R}, \mathbb{R})$ und es gelte $y'(x) = \alpha(x) y(x)$ $(x \in \mathbb{R})$.
\begin{displaymath}
  \exists x_0 \in \mathbb{R} : y(x_0) = 0.
  \quad \fbox{\rule{1in}{0pt}\rule[-0.5ex]{0pt}{4ex}} \quad
  y(x) = 0 \quad (x \in \mathbb{R}).
\end{displaymath}
Antwort: $\iff$

\section{Lineare Differentialgleichungen $n$-ter Ordnung mit konstanten Koeffizienten}
Partikuläre Lösungsansätze sind unter \url{https://www-user.tu-chemnitz.de/~peju/skripte/gdgl/Merkblatt_PL.pdf} zu finden.
Gute Videos mit Beispielen unter \url{https://www.youtube.com/channel/UCbE3PU7rYtS1PkmVmm6CfpA}

\subsection{Homogene DGL}
Aufgabe:
\begin{displaymath}
  y^{(iv)}(x) - 3y''(x) + 2y'(x) = 0
\end{displaymath}
Lösung:
\begin{displaymath}
  y(x) = a + (bx + c)e^x + de^{-2x} \quad (a, \dots, d \in \mathbb{R})
\end{displaymath}

Geben Sie das reelle Fundamentalsystem $y_1, y_2 : \mathbb{R} \to \mathbb{R}$ von
\begin{displaymath}
  y''(x) - y(x) = 0
\end{displaymath}
an, welches $y_1(0) = y_2(0)$ und $y_1'(0) = -y_2'(0) = 1$ erfüllt.\\
Lösung:
\begin{displaymath}
  y_1 = e^x, \quad y_2 = e^{-x}
\end{displaymath}

Die allgemeine reelle Lösung der Differentialgleichung $y''(x) + 4y'(x) + 4y(x) = 0$ lautet?
\begin{displaymath}
  y(x) = (Ax + B) e^{-2x} \quad (A,B \in \mathbb{R}).
\end{displaymath}

\subsection{Inhomogene DGL}
Aufgabe:
\begin{displaymath}
  y''(x) - 2y'(x) + 17y(x) = 17x^2 + x
\end{displaymath}
Homogene Lösung:
\begin{align*}
  \cp_y(\lambda) &= \lambda^2 - 2\lambda + 17\\
  \implies \lambda_{1/2} &= 1 \pm 4i\\
  \implies y_h &= e^x(a \cos(4x) + b \sin(4x))
\end{align*}
Mit der Ansatzfunktion $y_p = cx^2 + dx + e$ folgt mittels Koeffizientenvergleich:
\begin{displaymath}
  c = 1, \quad d = \frac{5}{17}, \quad e = -\frac{24}{289}
\end{displaymath}
Dadurch ergibt sich die allgemeine Lösung
\begin{displaymath}
  y(x) = e^x(a \cos(4x) + b \sin(4x)) + x^2 + \frac{5}{14}x - \frac{24}{289} \quad (a,b \in \mathbb{R}).
\end{displaymath}

Lösen Sie das folgende Anfangswertproblem und geben Sie ein offenes Intervall $I \subseteq \mathbb{R}$ an, auf dem die Lösung definiert ist.
\begin{displaymath}
  y'' - 5y' + 4y = e^{2x}, \quad y(0) = 1, \quad y'(0) = -1
\end{displaymath}
Homogene Lösung:
\begin{align*}
  \cp_y(\lambda) &= (\lambda - 4)(\lambda - 1)\\
  \implies \lambda_1 &= 4, \quad \lambda_2 = 1\\
  \implies y_h &= c_1 e^{4x} + c_2 e^{x}
\end{align*}
Mit der Ansatzfunktion $y_p = Ce^{2x}$ folgt mittels Koeffizientenvergleich:
\begin{displaymath}
  C = -\frac{1}{2}
\end{displaymath}
Dadurch ergibt sich die allgemeine Lösung
\begin{displaymath}
  y(x) = c_1 e^{4x} + c_2 e^{x} - \frac{1}{2}e^{2x}.
\end{displaymath}
Mit den Anfwangswerten folgt weiter $c_1 = -\frac{1}{2}$ und $c_2 = 2$, also
\begin{displaymath}
  y(x) = -\frac{1}{2} e^{4x} + 2 e^{x} - \frac{1}{2}e^{2x} \quad (I = \mathbb{R}).
\end{displaymath}

Berechnen Sie die allgemeine Lösung der Differentialgleichung
\begin{displaymath}
  2y''(x) - y'(x) = \sin(x).
\end{displaymath}
Homogene Lösung:
\begin{align*}
  \cp_y(\lambda) &= \lambda(2\lambda - 1)\\
  \implies \lambda_1 &= 0, \quad \lambda_2 = \frac{1}{2}\\
  \implies y_h &= c_1 + c_2 e^{\frac{1}{2} x}
\end{align*}
Mit der Ansatzfunktion $y_p = a \cos(x) + b\sin(x)$ folgt mittels Koeffizientenvergleich:
\begin{displaymath}
  a = \frac{1}{5}, \quad b = -\frac{2}{5}
\end{displaymath}
Dadurch ergibt sich die allgemeine Lösung
\begin{displaymath}
  y(x) = c_1 + c_2 e^{\frac{1}{2} x} + \frac{1}{5}\cos(x) - \frac{2}{5}\sin(x) \quad (c_1, c_2 \in \mathbb{R}).
\end{displaymath}

Berechnen Sie die allgemeine Lösung der Differentialgleichung
\begin{displaymath}
  -y''(x) + 4y'(x) - 4y(x) = 169 \sin(3x). 
\end{displaymath}
Homogene Lösung:
\begin{align*}
  \cp_y(\lambda) &= -(\lambda - 2)^2\\
  \implies \lambda_1 &= 2\\
  \implies y_h &= (c_1 x + c_1)e^{2x}
\end{align*}
Mit der Ansatzfunktion $y_p = A\sin(3x) + B\cos(3x)$ folgt mittels Koeffizientenvergleich:
\begin{displaymath}
  A = 5, \quad b = -12
\end{displaymath}
Dadurch ergibt sich die allgemeine Lösung
\begin{displaymath}
  y(x) = (c_1 x + c_2)e^{2x} + 5\sin(3x) - 12\cos(3x) \quad (c_1, c_2 \in \mathbb{R}).
\end{displaymath}

Bestimmen Sie die allgemeine reelle Lösung der folgenden Differentialgleichung.
\begin{displaymath}
  y'''(x) + y(x) = 1 + x^2
\end{displaymath}
Homogene Lösung:
\begin{align*}
  \cp_y(\lambda) &= \lambda^3 + 1\\
  \implies \lambda_1 &= 1, \quad \lambda_{2,3} = e^{\pm i \frac{\pi}{3}} = \cos\left(\frac{\pi}{3}\right) \pm i \sin\left(\frac{\pi}{3}\right) = \frac{1}{2} \pm i \frac{\sqrt{3}}{2}\\
  \implies y_h &= C_1 e^{-x} + e^{\frac{1}{2} x} (C_2 \cos\left(\frac{\sqrt{3}}{2}x\right) + C_3 \sin\left(\frac{\sqrt{3}}{2}x\right))
\end{align*}
Mit der Ansatzfunktion $y_p = A + Bx + Cx^2$ folgt mittels Koeffizientenvergleich:
\begin{displaymath}
  A = 1, \quad B = 0, \quad C = 1.
\end{displaymath}
Dadurch ergibt sich die allgemeine Lösung
\begin{displaymath}
  y(x) = C_1 e^{-x} + e^{\frac{1}{2} x} (C_2 \cos\left(\frac{\sqrt{3}}{2}x\right) + C_3 \sin\left(\frac{\sqrt{3}}{2}x\right)) + 1 + x^2 \quad (C_1, C_2, C_3 \in \mathbb{R}).
\end{displaymath}

Bestimmen Sie die allgemeine reelle Lösung der folgenden Differentialgleichung.
\begin{displaymath}
  y''(x) + 4y'(x) + 4y(x) = xe^{-2x}.
\end{displaymath}
Homogene Lösung:
\begin{align*}
  \cp_y(\lambda) &= (\lambda + 2)^2\\
  \implies \lambda_1 &= -2\\
  \implies y_h &= (Ax + B)e^{-2x}
\end{align*}
Mit der Ansatzfunktion $y_p = x^2(ax + b)e^{-2x}$ folgt mittels Koeffizientenvergleich:
\begin{displaymath}
  a = \frac{1}{6}, \quad b = 0.
\end{displaymath}
Dadurch ergibt sich die allgemeine Lösung
\begin{displaymath}
  y(x) = (Ax + B) e^{-2x} + \frac{x^3}{6}e^{-2x} \quad (A,B,x \in \mathbb{R}).
\end{displaymath}

\subsection{Wahr oder Falsch?}
Sei $a \in \mathbb{R}$ und seien $y_1, y_2 : \mathbb{R} \to \mathbb{R}$ Lösungen der Differentialgleichung $y'' + ay = 0$.
Dann ist auch $y_1 + y_2$ eine Lösung der Differentialgleichung.\\
Wahr.

Die Funktionen $e^x, e^{-x}, e^{-2x}$ bilden ein Fundamentalsystem von $y'' + 3y' + 2y = 0$.\\
Falsch.

% TODO: Grenzwerte, Ableitungen, Folgen, Reihen, Topologie (kompakt, offen,...), Zusammenhänge

\end{document}