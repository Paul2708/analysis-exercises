\section{Integration imm $\mathbb{R}^n$}
\subsection{Kreisfläche}
Es sei $B:= \{(x,y) \in \mathbb{R}^2 : x^2 + y^2 \leq e - 1\}$.
Dann gilt
\begin{displaymath}
  \int_{B} \frac{1}{1 + x^2 + y^2}d(x,y) = \pi
\end{displaymath}
Lösung:
Durch einen Übergang in Polarkoordinaten ergibt sich
\begin{displaymath}
  \int_{B} \frac{1}{1 + x^2 + y^2} d(x,y) = \int_{0}^{2\pi} \int_{0}^{\sqrt{e - 1}} \frac{r}{1 + r^2} drd\varphi = 2\pi \frac{1}{2}[\log(1 + r^2)]_{0}^{\sqrt{e-1}} = \pi \log(e) - 0 = \pi.
\end{displaymath}

Es sei $B = \{(x,y) \in \mathbb{R}^2 : x,y \geq 0, x^2 + y^2 \leq 1\}$.
Berechnen Sie
\begin{displaymath}
  \int_B \frac{2}{x^2 + y^2 + 1} \diff (x,y).
\end{displaymath}
Lösung:
Der Übergang in Polarkoordinaten liefert
\begin{displaymath}
  \int_B \frac{2}{x^2 + y^2 + 1} \diff (x,y) = \int_0^{\frac{\pi}{2}} \int_0^1 \frac{2r}{r^2 + 1} \diff r \diff \varphi = \dots = \frac{\pi \log 2}{2}.
\end{displaymath}

Es se $M := \{(x,y) \in \mathbb{R}^2 : x^2 + y^2 \leq 2, |x| > y\}$.
Berechnen Sie
\begin{displaymath}
  \int_M (x^2 + y^2) \diff (x,y).
\end{displaymath}
Der Übergang in Polarkoordinaten liefert
\begin{displaymath}
  \int_M (x^2 + y^2) \diff (x,y) = \int_{-\frac{5}{4}\pi}^{\frac{1}{4} \pi} \int_0^{\sqrt{2}} r^2 \cdot r \diff r \diff \varphi = \dots = \frac{3}{2} \pi.
\end{displaymath}

Das Volumen der Menge $B := \{(x,y,z) \in \mathbb{R}^3 : 0 \leq z \leq \pi, x^2 + y^2 \leq \sin^2(z)\}$.
\begin{displaymath}
  |B| = \int_0^{\pi} \int_0^{2\pi} \int_0^{\sin(z)} r \diff r \diff \varphi \diff z = \dots \text{ partielle Integration } \dots = \frac{\pi^2}{2}.
\end{displaymath}

\subsection{Satz von Fubini}
Sei $B := \{(x,y,z) \in \mathbb{R}^3 : 0 \leq x \leq 1, x^2 \leq y \leq \sqrt{x}, 0 \leq z \leq 1\}$.
Berechnen Sie das folgende Integral:
\begin{displaymath}
  \int_B (xyz) d(x,y,z)
\end{displaymath}
Lösung:
\begin{displaymath}
  \int_B (xyz) d(x,y,z) = \int_0^1 \int_0^1 \int_{x^2}^{\sqrt{x}} (xyz) \diff y \diff x \diff z = \dots = \frac{1}{24}.
\end{displaymath}

Sei $D = \{(x,y) \in \mathbb{R}^2 : y \in [0,2], y \leq x \leq y^2 + 1\}$.
Berchnen Sie den Flächeninhalt $|D|$.
\begin{displaymath}
  |D| = \int_D 1 \diff (x,y) = \int_0^2 \int_y^{y^2 + 1} 1 \diff x \diff y = \frac{8}{3}.
\end{displaymath}

Sei $C := [0,1] \times [0,2]$.
\begin{displaymath}
  \int_C (y - \sin(x)) \diff(x,y) = \int_0^1 \int_0^2 (y - \sin(x)) \diff y \diff x = \dots = 2 \cos(1)
\end{displaymath}

\begin{displaymath}
  \int_{[0,1]^2} (xy + y^2) \diff(x,y) = \int_0^1 \int_0^1 (xy + y^2) \diff y \diff x = \dots = \frac{7}{12}
\end{displaymath}

\subsection{Normalbereich und Cavalieri}
Es sei $M := \{(x,y) \in \mathbb{R}^2 : x^2 \geq y \geq 2x^2 - 4, y \geq 0\}$.
\begin{displaymath}
  \int_M 1 \diff (x,y) = \int_0^2 \int_{2x^2 - 4}^{x^2} 1 \diff x \diff y = \frac{16}{3}
\end{displaymath}

Es sei $\triangle \subseteq \mathbb{R}^2$ das Dreieck mit den Ecken $(0,0), (1,0)$ und $(0,1)$.
Berechnen Sie
\begin{displaymath}
  \int_{\triangle} x^2e^y \diff (x,y) = \int_0^1 \int_0^{1 - x} x^2 e^y \diff y \diff x = \dots = 2e - \frac{16}{3}.
\end{displaymath}
Hinweis: partielle Integration

Es sei $B := \{(x,y) \in \mathbb{R}^2 : x^2 - 2x \leq y \leq x\}$.
Berechnen Sie $\vol(B)$.\\
Lösung:
\begin{displaymath}
  \vol(B) = \int_B 1 \diff(x,y) = \int_0^3 \int_{x^2 - 2x}^{x} 1 \diff y \diff x = \dots = \frac{9}{2}.
\end{displaymath}

Es sei $B = \{(x,y) \in \mathbb{R}^2 : x \geq 0, \frac{1}{4}x^2 - 1 < y < 2 - x\}$
Dann gilt
\begin{displaymath}
  \int_B 1 \diff(x,y) = \int_0^2 \int_{\frac{1}{4}x^2 - 1}^{2 - x} 1 \diff y \diff x = \dots = \frac{10}{3}.
\end{displaymath}

\subsection{Wahr oder Falsch?}
Das Volumen der Einheitskugel im $\mathbb{R}^3$ beträgt $\frac{4\pi}{3}$.\\
Wahr.

Sei $g \in C[0,1]$ und $Q$ der durch
\begin{displaymath}
  Q := \{(x,y,z) \in \mathbb{R}^3 : y^2 + z^2 \leq g^2(x), x \in [0,1]\}
\end{displaymath}
definierte Rotationskörper.
Dann ist das Volumen von $Q$ gegeben durch $|Q| = \pi \int_{0}^{1} g^2(x)dx$.\\
Wahr.

Sei $D = \{(x,y,z) \in \mathbb{R}^3 : z \in [0,1], x^2 + y^2 \leq 1\}$.
Dann gilt mit Zylinderkoordinaten:
\begin{displaymath}
  \int_D (x^2 + y^2)e^x d(x,y,z) = \int_0^1 \int_0^{2\pi} \int_0^1 r^2 e^z \diff r \diff \varphi \diff z
\end{displaymath}
Falsch.

Sei $f : [0,3]^2 \to [-2,2]$ stetig.
Dann gilt:
\begin{displaymath}
  |\int_{[0,3]^2} f(x) \diff x | \leq 18
\end{displaymath}
Wahr.

Sei $f : [0,1] \times [0,1] \to \mathbb{R}$ stetig.
Dann gilt
\begin{displaymath}
  \int_0^1 \int_0^y f(x,y) \diff x \diff y = \int_0^1 \int_0^x f(x,y) \diff y \diff x.
\end{displaymath}
Falsch.

Sei $a < b$, seien $f,g : [a,b] \to \mathbb{R}$ stetig mit $f(x) \leq g(x)$ für alle $x \in [a,b]$ und sei $A \subseteq \mathbb{R}^2$ gegeben durch
\begin{displaymath}
  A := \{(x,y) \in [a,b] \times \mathbb{R} : f(x) \leq y \leq g(x)\}.
\end{displaymath}
Der Flächeninhalt von $A$ berechnet sich gemäß der Formel
\begin{displaymath}
  |A| = \int_a^b g(x) \diff x - \int_a^b f(x) \diff x.
\end{displaymath}
Wahr.

\subsection{Beziehungen}
Es sei $f : [0,1]^2 \to \mathop{R}$ stetig.
\begin{displaymath}
  f(x,y) \geq 0 \quad ((x,y) \in [0,1]^2).
  \quad \fbox{\rule{1in}{0pt}\rule[-0.5ex]{0pt}{4ex}} \quad
  \int_{[0,1]^2} f(x,y) \diff (x,y) \geq 0.
\end{displaymath}
Antwort: $\implies$