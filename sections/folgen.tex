\section{Folgen}
\subsection{Grenzwerte}
\begin{displaymath}
  \lim_{n \to \infty} {\sqrt{n^2 + n} - 3n} = - \infty
\end{displaymath}
Idee: 3. Binomische Formel

\begin{displaymath}
  \lim_{n \to \infty} \frac{(n+1)^n \cdot (2n+1)^3}{n^n} = 8e
\end{displaymath}
Idee: Ausdividieren und $\frac{a^n}{b^n} = \left( \frac{a}{b}\right) ^n$ ausnutzen.

\begin{displaymath}
  \lim_{n \to \infty} (\sqrt{3n-2} - \sqrt{3n+1}) = 0
\end{displaymath}
Idee: 3. Binomische Formel

\begin{displaymath}
  \lim_{n \to \infty} \left(1 + \frac{a}{b \cdot n + k}\right)^{cn} = e^{c \cdot \frac{a}{b}}
\end{displaymath}
Idee: Kein Plan. Ist aber so

\begin{displaymath}
  \liminf_{n \to \infty} \left(\left(1 + \frac{2}{n}\right)^n \cdot (-1)^n \right) = -e^2
\end{displaymath}
Idee: Häufungswerte bestimmen oder Limes reinziehen

\begin{displaymath}
  \lim_{n \to \infty} \left(1 - \frac{1}{n^2}\right)^n = 1
\end{displaymath}
Idee: 3. Binomische Formel und Grenzwertsatz benutzen

\subsection{Rekursive Folgen}
Sei $a_1 := \frac{3}{2}$ und $a_{n+1} = \frac{3}{4-a_n}$.
\begin{enumerate}
    \item Zeigen Sie mit vollständiger Induktion $1 \leq a_{n+1} \leq a_n \leq \frac{3}{2}$.
    \item Zeigen Sie, dass die Folge $a_n$ konvergiert und berechnen Sie ihren Grenzwert.
\end{enumerate}

Antwort:
\begin{enumerate}
    \item
    \begin{enumerate}
        \item Betrachte $1 \leq a_n \leq \frac{3}{2}$ und $a_{n+1} \leq a_n$.
        \item Induktionsanfang ist klar.
        \item Induktionsschritt mit $a_{n+1} = \frac{3}{4 - a_n}$ anfangen (bzw. $a_{n+2} = \frac{3}{4 - a_{n+1}}$) und IV einsetzen
    \end{enumerate}
    \item
    \begin{enumerate}
        \item Mit 1. folgt, dass die Folge monton fallend und beschränkt ist.
        Daher konvergiert sie auch.
        \item Für den Grenzwert $a := \lim_{n \to \infty} a_n$ gilt:
        \begin{displaymath}
          a = \frac{3}{4 - a}
          \iff (a-3)(a-1) = 0
          \iff a = 1.
        \end{displaymath}
    \end{enumerate}
\end{enumerate}

Sei $a \in (-1,1)$, $b \in \mathbb{R}$, $x_0 \in \mathbb{R}$ beliebig. Für $n \in \mathbb{N}_0$ definiere
\begin{displaymath}
  x_{n+1} := ax_n + b.
\end{displaymath}
Zeigen Sie:
\begin{enumerate}
    \item Es existiert genau ein $x \in \mathbb{R}$ mit $x = ax+b$. Bestimmen Sie dieses.
    \item Für alle $n \in \mathbb{N}$ gilt $x_n - \frac{b}{1-a} = a^n\left(x_0 - \frac{b}{1-a}\right)$.
    \item $\lim_{n \to \infty} x_n = \frac{b}{1-a}$.
\end{enumerate}

Antwort:
\begin{enumerate}
    \item
    \begin{enumerate}
        \item $x$ ausklammern, durch $1-a$ teilen.
        \item Mit den Äquivalenzumformungen folgt die Eindeutigkeit und die Existenz.
    \end{enumerate}
    \item
    \begin{enumerate}
        \item Mithilfe vollständiger Induktion.
    \end{enumerate}
    \item
    \begin{enumerate}
        \item Benutze den Satz: $|a_n - a| \leq b_n$ und $b_n \to 0$, so gilt $a_n \to a$.
        \item Betrachte $|x_n - \frac{b}{1-a}|$ und benutze die 2.
        \item Mit $|a| < 1$ folgt die Behauptung.
    \end{enumerate}
\end{enumerate}

\subsection{Wahr oder falsch?}
Besitzt eine Folge reeller Zahlen genau einen Häufungspunkt in $\mathbb{R}$, so ist diese Folge konvergent.\\
Falsch.\\
Betrachte $a_n = (0,1,0,2,0,3,\dots)$. $0$ ist offensichtlich ein Häufungspunkt, $a_n$ ist aber nicht mal beschränkt.

Seien $a_n$, $b_n$ beschränkte Folgen reeller Zahlen.
\begin{displaymath}
  \liminf_{n \to \infty} (a_n - b_n) = \liminf_{n \to \infty} a_n - \liminf_{n \to \infty} b_n
\end{displaymath}
Falsch.\\
Gegenbeispiel: $a_n = 1$, $b_n = (-1)^n$ oder $a_n = (0,1,0,1,\dots)$, $b_n = (1,0,1,0,\dots)$.

Sei $a_n$ monoton und nach oben beschränkt.
Dann ist $a_n$ konvergent.\\
Falsch.\\
Die Folge kann auch monton fallend sein.

Sei $a_n$ konvergent mit $a := \lim_{n \to \infty} a_n$.
Dann konvergiert jede Teilfolge gegen $a$.\\
Wahr.\\
Bew.: $\varepsilon$-Umgebung

Eine beschränkte Folge hat einen Häufungswert.\\
Wahr.\\
Bew.: Bolzano-Weierstraß.

Jede reelle Zahlenfolge enthält eine monotone Teilfolge.\\
Wahr.\\
Bew.: Mit Indizes spielen und Definition von niedrig ($m$ heißt niedrig $\iff \forall n \geq m : a_n \geq a_m$).

Sei $a_n$ eine beschränkte Folge.
Dann hat $a_n$ eine monoton fallende Teilfolge.\\
Falsch.\\
Die Teilfolge kann auch monoton steigend sein.

\subsection{Häufungswerte}
Bestimmen Sie die Menge der Häufungswerte der Folge $a_n := e^{(-1)^{n}n-n} \sin\left(\frac{\pi n}{2}\right)$.\\
Antwort:
Für alle $k \in \mathbb{N}$ gilt:
\begin{align*}
  a_{2k} &= e^{(-1)^{2k}2k-2k}\sin\left(\frac{2k\pi}{2}\right) = e^0 \sin(k\pi) = 0\\
  a_{2k+1} &= e^{(-1)^{2k+1}(2k+1)-(2k+1)}\sin\left(\frac{(2k+1)\pi}{2}\right) = e^{-4k-2} \sin\left(\frac{(2k+1)\pi}{2}\right) \to 0 (k \to \infty).
\end{align*}
Da $e^{-4k-2}$ gegen $0$ geht und der Sinus beschränkt ist.

Bestimmen Sie die Menge der Häufungswerte der Folge $a_n := \frac{\left(1+\sin\left(\frac{\pi}{2}n\right)\right)^n}{2}$.\\
Antwort: Für alle $k \in \mathbb{N}$ gilt:
\begin{align*}
  a_{2k} &= \frac{1}{2} \left(1+\sin\left(\frac{\pi}{2}2k\right)\right)^{2k} = \frac{1}{2} \left(1+\sin\left(\pi k\right)\right)^{2k} = \frac{1}{2} (1 + 0)^{2k} = \frac{1}{2}\\
  a_{4k-1} &= \frac{1}{2} \left(1+\sin\left(\frac{\pi}{2}(4k-1)\right)\right)^{4k-1} = \frac{1}{2} (1 - 1)^{4k-1} = 0\\
  a_{4k+1} &= \frac{1}{2} \left(1+\sin\left(\frac{\pi}{2}(4k+1)\right)\right)^{4k+1} = \frac{1}{2} (1 + 1)^{4k+1} \to \infty \text{ } (k \to \infty) 
\end{align*}
Beachte, dass jede Teilfolge betrachtet wurde.

\subsection{Beziehungen}
Sei $a_n$ eine Folge.
\begin{displaymath}
  a_n \to 0 \text{ für } n \to \infty \text{. } \fbox{\rule{1in}{0pt}\rule[-0.5ex]{0pt}{4ex}} \text{ $a_n$ ist beschränkt.}
\end{displaymath}
Mögliche Beziehungen:
$\implies$, $\Leftarrow$, $\iff$, k.B.\\
Antwort: $\implies$