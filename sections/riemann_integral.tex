\section{Das Riemann-Integral}

\subsection{Ableitungen}
Berechnen Sie die Ableitung von
\begin{displaymath}
  g(x) = \int_{0}^{x^2} \left(\int_{0}^{t} \cos(t)e^s ds\right)dt.
\end{displaymath}
Idee: $g(x) = F(x^2) - F(0)$\\
Lösung: $g'(x) = 2x \cos(x^2) \left(e^{x^2} - 1\right)$

\subsection{Integrale ausrechnen}
\begin{displaymath}
  \int_{0}^{1} xe^{x + 1} = e
\end{displaymath}
Idee: $e$ heraus ziehen und partielle Integration mit $f' = e^x$ und $g = x$

\begin{displaymath}            
  \int_{1}^{\pi + 1} t\sin(t - 1)dt = \pi + 2                                   
\end{displaymath}
Idee: (Index-Shift), partielle Integration mit $f' = \sin(t - 1)$ und $g = t$

\begin{displaymath}
  \int_{1}^{\infty} \frac{\log^2(x)}{x^2} dx = 2
\end{displaymath}
Idee: Zwei mal partiell integrieren mit $f' = \frac{1}{x^2}$ und $g = \log^2(x)$ oder substituiere $t := \log(x)$, also auch $x = e^t$

\begin{displaymath}
  \int_{0}^{1} x^{-\frac{1}{2}} dx = 2
\end{displaymath}
Idee: Grenzwert $\lim_{n \to 0+}$ und normal lösen

\begin{displaymath}
  \int_{0}^{1} x^3e^{x^2} dx = \frac{1}{2}
\end{displaymath}
Idee: Partielle Integration mit $f' = e^{x^2}2x$ und $g = \frac{1}{2}x^2$ oder substituiere $t := x^2$

\begin{displaymath}
  \int_{\frac{\pi}{2}}^{\pi} x \cdot \sin(x) dx = \pi - 1
\end{displaymath}
Idee: Partielle Integration mit $f' = \sin(x)$ und $g = x$

\begin{displaymath}
  \int_{-e}^{-1} \frac{1}{x} dx = -1
\end{displaymath}
Idee: Punktsymmetrie von $\frac{1}{x}$ bzw. Stammfunktion des $\log(\cdot)$ benutzen

\begin{displaymath}
  \int_{0}^{3\pi} x \sin(x) dx = 3\pi
\end{displaymath}
Idee: Partielle Integration mit $f' = \sin(x)$ und $g = x$

\begin{displaymath}
  \int_{e^4}^{\infty} \frac{1}{x[\log\sqrt{x}]^4}dx = \frac{1}{12}
\end{displaymath}
Idee: Substituiere $t := \log\sqrt{x}$ mit $dt = \frac{1}{2x}dx$ und bilde den Grenzwert

\begin{displaymath}
  \int_{1}^{\infty} \cos\left(\frac{1}{x}\right) \frac{1}{x^2} dx = \sin(1)
\end{displaymath}
Idee: Substituiere $t := \frac{1}{x}$ mit $dt = -\frac{1}{x^2}dx$

\begin{displaymath}
  \int_{0}^{\pi} \sin^2(x) e^x dx = \frac{2}{5} \left(e^{\pi} - 1\right)
\end{displaymath}
Idee: Zweifache partielle Integration mit $f' = e^x$, $g = \sin^2(x)$ und $f' = e^x$, $g = \sin(x)\cos(x)$ mit dem Wissen von $\cos^2(x) = 1 - \sin^2(x)$

\begin{displaymath}
  \int_{1}^{\infty} \frac{1}{x\sqrt{x - 1}} dx = \pi
\end{displaymath}
Idee: Substituiere $t := \sqrt{x - 1}$, also $x = t^2 + 1$ mit $dt = \frac{1}{2\sqrt{x - 1}}dx$ und $\left(\arctan(x)\right)' = \frac{1}{1 + x^2}$ und $\lim_{n \to \infty} \arctan(n) = \frac{\pi}{2}$

\begin{displaymath}
  \int_{-1}^{2} x^2 e^{x^3 + 1} dx = \frac{1}{3} \left(e^9 - 1\right)
\end{displaymath}
Idee: Substituiere $t := x^3$.

\begin{displaymath}
  \int_{1}^{e} x \log(x) dx = \frac{1}{4} \left(e^2 + 1\right)
\end{displaymath}
Idee: Partielle Integration mit $f' = x$ und $g = \log(x)$

\begin{displaymath}
  \int_{0}^{1} t \cdot \arctan(t) dt = \frac{\pi}{4} - \frac{1}{2}
\end{displaymath}
Idee: Partielle Integration mit $f' = t$ und $g = \arctan$ mit $\left(\frac{t^2 + 1}{2}\right)' = f$

\begin{displaymath}
  \int_{0}^{\infty} xe^{-tx} dx = \frac{1}{t^2}
\end{displaymath}
Idee: Partielle Substitution mit $f' = e^{-tx}$ und $g = x$. Grenzwertbetrachtung mit l'Hospital

\subsection{Wahr oder Falsch?}
Die Funktion
\begin{displaymath}
  f: (0,\infty) \to \mathbb{R}, f(x) = \int_{0}^{x} e^{-t^2}dt
\end{displaymath}
ist streng monoton wachsend.\\
Wahr.\\
Begründung: $e^{-t^2} > 0$

Sei $f$ eine nicht Riemann-integrierbare Funktion auf $[a,b]$, $a,b \in \mathbb{R}$, $a < b$.
Dann ist $f$ nicht stetig.\\
Wahr.\\
Begründung: Jede stetige Funktion ist auch integrierbar.

Sei $f:[a,b] \to \mathbb{R}$ eine Riemann-integrierbar.
Dann besitzt $f$ auch eine Stammfunktion.\\
Falsch.\\
Bemerkung im Skript.

Das uneigentliche Riemann-Integral $\int_{0}^{\infty} f(x)dx$ existiere.
Dann gilt $\lim\limits_{x \to \infty} f(x) = 0$.\\
Falsch.\\

Jede differenzierbare Funktion $f:[0,1] \to \mathbb{R}$ ist Riemann-integrierbar.\\
Wahr.\\
Beweis: Jede differenzierbare Funktion ist stetig und jede stetige Funktion ist Riemann-integrierbar.

Das Produkt zweier Riemann-integrierbaren Funktionen ist wieder Riemann-integrierbar.\\
Wahr.\\
Beweis: Satz aus der Vorlesung

Sei $f: \mathbb{R} \to \mathbb{R}$ stetig und $a \in \mathbb{R}$.
Dann gilt
\begin{displaymath}
  \lim_{\varepsilon \to 0} \int_{a}^{a + \varepsilon} f(x)dx = 0.
\end{displaymath}
Wahr.\\

Sei $f : [0, \infty)\to [0,\infty)$ stetig und es gelte $\int_{0}^{x} f(t)dt \leq 100$ für alle $x \in [0, \infty)$.
Dann ist $f$ uneigentlich Riemann-integrierbar auf $[0,\infty)$.
Wahr.\\

Sei $f_n$ eine Folge in $R([0,1])$ und $f_n$ konvergiert für $n \to \infty$ auf $[0,1]$ punktweise gegen $f: [0,1] \to \mathbb{R}$.
Dann gilt
\begin{displaymath}
  \lim_{n \to \infty} \int_{0}^{1} f_n(x)dx = \int_{0}^{1} f(x)dx.
\end{displaymath}
Falsch.\\
Begründung: Das gilt, falls $f_n$ gleichmäßig konvergiert (und nicht nur punktweise).

Ist $f \in R([0,1])$, so besitzt $f$ eine Stammfunktion.\\
Falsch.\\
Bemerkung in der Vorlesung.

Es sei $f \in C([0,1])$.
Dann ist $f \in R([0,1])$.\\
Wahr.\\
Begründung: Jede stetige Funktion ist Riemann-integrierbar.

Sind $f,g \in R([1,a])$ für alle $a > 1$ und gilt $0 \leq f(x) \leq g(x)$ für alle $x \in [1,\infty)$, dann gilt:
\begin{displaymath}
  \int_{1}^{\infty} f(x) dx \text{ divergent. } \implies \int_{1}^{\infty} g(x) dx \text{ divergent. }
\end{displaymath}
Wahr.

Sind $f,g \in R([a,b])$ für $a,b \in \mathbb{R}$, $a < b$, und ist $f(x) \leq g(x)$ für alle $x \in [a,b]$, dann gilt:
\begin{displaymath}
  \int_{a}^{b} f(x) dx \leq \int_{a}^{b} g(x) dx.
\end{displaymath}
Wahr.

Seien $a,b \in \mathbb{R}$, $a < b$ und $f \in C([a,b])$ und $f(x) \geq 0$.
Dann gilt
\begin{displaymath}
  \int_{a}^{b} f(t) dt = 0 \implies f(x) = 0 \text{ für alle } x \in [a,b].
\end{displaymath}
Wahr.

\subsection{Beziehungen}
Es sei $f: \mathbb{R} \to \mathbb{R}$ eine Funktion.
\begin{displaymath}
  f \text{ ist stetig. } \fbox{\rule{1in}{0pt}\rule[-0.5ex]{0pt}{4ex}} \text{ $f$ besitzt eine Stammfunktion.}
\end{displaymath}
Lösung: $\Rightarrow$ (Hauptsatz der Differential- und Integralrechnung)