\section{Unendliche Reihen}
\subsection{Reihenwerte}
\begin{displaymath}
  \sum_{k = 0}^{\infty} \frac{(-1)^k}{2^k} = \frac{2}{3}
\end{displaymath}
Idee: Potenzgesetz anwenden und als geometrische Reihe auffassen.

\begin{displaymath}
  \sum_{k = 1}^{\infty} \left(\frac{1}{k+1}-\frac{1}{k}\right) = -1
\end{displaymath}
Idee: Teleskopsumme

\begin{displaymath}
  \sum_{k = 0}^{\infty} \frac{(-1)^k}{3^k} = \frac{3}{4}
\end{displaymath}
Idee: Potenzgesetz anwenden und als geometrische Reihe auffassen.

\begin{displaymath}
  \sum_{n = 1}^{\infty} \frac{2^n}{3^{n+1}} = \frac{2}{3}
\end{displaymath}
Idee: Ausklammern, Index-Shift, geometrische Reihe

\subsection{Wahr oder Falsch?}
Sei $a_n$ eine beschränkte Folge reeller Zahlen.
Falls $a_n \cdot a_{n+1} < 0$ für alle $n \in \mathbb{N}$, dann konvergiert $\sum\limits_{n = 1}^{\infty} a_n$.\\
Falsch.\\
Betrachte das Gegenbeispiel $a_n = (-1)^n$.

Sei $a_n$ eine Folge mit $\sqrt[n]{|a_n|} < 1$ für alle $n \in \mathbb{N}$. Dann konvergiert die Reihe $\sum\limits_{n = 1}^{\infty} a_n$.\\
Falsch.\\
Falsche Anwendung des Wurzelkriteriums. Der Limes Superior muss $< 1$ sein, nicht die Folgenglieder.

Sind $a_n \in \mathbb{R}$, $n \in \mathbb{N}$, und gilt $|a_n| \leq \frac{1}{n}$ für alle $n \in \mathbb{N}$, dann ist die Reihe $\sum_{n = 1}^{\infty} (-1)^n a_n$ konvergent.\\
Falsch.\\
Betrachte $a_n := \left((-1)^n \frac{1}{n}\right)$. Die Folge erfüllt die Bedingung, aber $\sum_{n = 1}^{\infty} (-1)^n a_n = \sum_{n = 1}^{\infty} \frac{1}{n}$ ist divergent.

Es gibt absolut konvergente Reihen, die nicht konvergieren.\\
Falsch.\\
Nach der VL gilt
\begin{displaymath}
  \sum_{n = 1}^{\infty} a_n \text{ ist absolut konvergent. } \implies \sum_{n = 1}^{\infty} a_n \text{ ist konvergent. }
\end{displaymath}

Es gibt konvergente Reihen $\sum_{n = 0}^{\infty} a_n$ derart, dass das Chauchy-Produkt der Reihe mit sich selbst divergiert.\\
Wahr.\\
Betrachte die Reihe $\sum_{k = 1}^{\infty} \frac{(-1)^{k+1}}{\sqrt{k}}$.
Mit dem Leibniz-Kriterium folgt die Konvergenz.
Das Chauchy-Produkt der Reihe mit sich selbst ist $\sum_{m = 1}^{\infty} c_m$ mit $c_m = (-1)^m \sum_{j = 1}^{m - 1} \frac{1}{\sqrt{j} \sqrt{m-j}}$.
Da $|c_m| \geq 1$ für alle $m \geq 2$ ist $\sum_{m = 1}^{\infty} c_m$ also divergent.

\subsection{Beziehungen}
Sei $a_n$ eine reelle Folge.
\begin{displaymath} 
  \sum_{n = 1}^{\infty} a_n^2 \text{ konvergiert. } \fbox{\rule{1in}{0pt}\rule[-0.5ex]{0pt}{4ex}} \text{ } \sum_{n = 1}^{\infty} a_n \text{ konvergiert.}
\end{displaymath}
Antwort: keine Beziehung

\begin{displaymath}
  \sum_{n = 1}^{\infty} a_n^2 \text{ konvergiert. } \fbox{\rule{1in}{0pt}\rule[-0.5ex]{0pt}{4ex}} \text{ } \sum_{n = 1}^{\infty} a_n \text{ konvergiert absolut.}
\end{displaymath}
Antwort: $\Leftarrow$

\begin{displaymath}
  \sum_{n = 1}^{\infty} a_n \text{ konvergiert. } \fbox{\rule{1in}{0pt}\rule[-0.5ex]{0pt}{4ex}} \text{ } \limsup_{n \to \infty} \sqrt[n]{|a_n|} \leq 1
\end{displaymath}
Antwort: $\Rightarrow$

\begin{displaymath}
  \sum_{n = 1}^{\infty} a_n \text{ konvergiert absolut. } \fbox{\rule{1in}{0pt}\rule[-0.5ex]{0pt}{4ex}} \text{ } \limsup_{n \to \infty} \sqrt[n]{|a_n|} \leq 1
\end{displaymath}
Antwort: $\Rightarrow$

\begin{displaymath}
  \sum_{n = 1}^{\infty} a_n \text{ konvergiert absolut. } \fbox{\rule{1in}{0pt}\rule[-0.5ex]{0pt}{4ex}} \text{ } a_n \to 0 \text{ für } n \to \infty \text{.}
\end{displaymath}
Antwort: $\Rightarrow$

\begin{displaymath}
  a_n \to 0 \text{ für } n \to \infty \text{.} \fbox{\rule{1in}{0pt}\rule[-0.5ex]{0pt}{4ex}} \text{ } a_n \text{ ist beschränkt.}
\end{displaymath}
Antwort: $\Rightarrow$