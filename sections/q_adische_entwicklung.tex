\section{q-adische Entwicklung}
Sei $0,121212\dots$ die $3$-adische Entwicklung einer Zahl $a \in \mathbb{R}$. Bestimmen Sie $m, n \in \mathbb{N}$ mit $a = \frac{m}{n}$.\\
Lösung: $a = \frac{5}{8}$\\
Idee: Ausgehend von der q-adschen Darstellung kann man zwei Summen betrachten, Index-Shifts, ausklammern, geometrische Reihe 