\section{Fouriertransformation}

\subsection{Verschiedenes}
Es sei $f : \mathbb{R} \to \mathbb{R}$ mit
\begin{displaymath}
  f(t) = 
  \begin{cases}
    t^3,& -1 \leq t \leq 2\\
    0,& \text{sonst}
  \end{cases}    
\end{displaymath}
und $\hat{f}$ bezeichne ihre Fouriertransformierte. 
Dann gilt
\begin{displaymath}
  CH - \int_{-\infty}^{\infty} \hat{f}(s) e^{ist} \diff s = 
  \begin{cases}
    0,& t < -1,\\
    -\frac{1}{2},& t = -1,\\
    t^3,& -1 < t < 2,\\
    4,& t = 2,\\
    0,& t > 2.\\
  \end{cases}
\end{displaymath}

\subsection{Beziehungen}
Es sei $f : \mathbb{R} \to \mathbb{R}$.
\begin{displaymath}
  CH - \int_{-\infty}^{\infty} f(x) \diff x \text{ existiert. }
  \quad \fbox{\rule{1in}{0pt}\rule[-0.5ex]{0pt}{4ex}} \quad
  f \text{ ist uneigentlich Riemann-integrierbar.}
\end{displaymath}
Antwort: $\Leftarrow$