\section{Funktionsfolgen und -reihen}
\subsection{Grenzfunktion und gleichmäßige Konvergenz}
Sei die Funktionsfolge $f_n$ definiert durch
\begin{displaymath}
  f_n : \mathbb{R} \to \mathbb{R} \text{ mit } f_n(x)=
  \begin{cases}
    1        & \text{für } x < 0\\
    \cos(nx) & \text{für } 0 \leq x \leq \frac{\pi}{n}\\
    -1       &\text{für} x > \frac{\pi}{n} 
  \end{cases}
\end{displaymath}
Berechnen Sie die punktweise Grenzfunktion und bestimmen Sie, ob die Funktionsfolge auch gleichmäßig konvergiert.\\
Idee: Fallunterscheidung $x < 0$, $x = 0$ und $x > 0$ (betrachte hier, dass $\frac{\pi}{n}$ eine Nullfolge ist), Stetigkeit der Funktionen prüfen.\\
Lösung: $f(x) = \begin{cases} 1 & \text{für } x \leq 0\\ -1 & \text{für } x > 0\end{cases}$ und keine gleichmäßige Konvergenz.

Untersuchen Sie die Funktionsfolge $f_n : \mathbb{R} \to \mathbb{R}, f_n(x) = n \cdot \sin\left(\frac{x}{n}\right)$ auf punktweise und gleichmäßige Konvergenz.\\
Idee: Fallunterscheidung $x = 0$ und $x \neq 0$ (hier umformen auf $\frac{\sin(y)}{y} \xrightarrow{y \to \infty}$ 1).
Betrachte Folge $x_n = n\pi$ und die Differenz beider Funktionen (oder, dass $f$ nicht beschränkt ist)\\
Lösung: Punktweise konvergent mit $f : \mathbb{R} \to \mathbb{R}, f(x) = x$. Nicht gleichmäßig konvergent.

Für $n \in \mathbb{N}$ betrachten wir die Funktion
\begin{displaymath}
  f_n: [0,1] \to \mathbb{R}; x \mapsto n^2x(1-x)^n.
\end{displaymath}
Zeigen Sie, dass die Folge $f_n$ auf ganz $[0,1]$ punktweise gegen $f : [0,1] \to \mathbb{R}$ konvergiert und bestimmen Sie die Grenzfunktion.
Konvergiert die Reihe $f_n$ sogar gleichmäßig auf $[0,1]$?\\
Idee: Nutze $\lim_{n \to \infty} n^m \cdot q^n = 0$ mit $q \in [0,1)$. Und teste die Folge $x_n = \frac{1}{n}$.\\
Lösung: $f(x) = 0$ und $f_n$ konvergiert nicht gleichmäßig

Betrachten Sie die Funktionsfolge $f_n:[0,1] \to \mathbb{R}$ definiert durch
\begin{displaymath}
  f(x) := 
  \begin{cases}
    2nx,& \text{falls } x \in [0,\frac{1}{2n}),\\
    2 - 2nx,& \text{falls } x \in [\frac{1}{2n}, \frac{1}{n}],\\
    0,& \text{falls } x \in (\frac{1}{n}, 1].
  \end{cases}
\end{displaymath}
Prüfen Sie die Funktionsfolge auf punktweise und gleichmäßige Konvergenz.\\
Idee: Fallunterscheidung $x = 0$, $x \neq 0$ (hier gilt $\frac{1}{n} < x$ für ffa $n \in \mathbb{N}$.
Betrachte die Folge $x_n = \frac{1}{2n}$.\\
Lösung: $f(x) = 0$ und keine gleichmäßige Konvergenz.

Konvergiert die Funktionenreihe $\sum_{k = 1}^{\infty} \frac{4 \sin(kx)}{k^2}$ gleichmäßig auf $\mathbb{R}$?\\
Idee: Folge abschätzen. Zeigen, dass die abgeschätzte Reihe konvergiert. $\implies$ Kriterium von Weierstraß\\
Lösung: Ja

Untersuchen Sie die Funktionenfolge $f_n$ definiert durch
\begin{displaymath}
  f_n : \mathbb{R} \to \mathbb{R}, f_n(x) := \cos\left(\frac{x}{n}\right)
\end{displaymath}
auf punktweise und gleichmäßige Konvergenz.\\
Idee: Stetigkeit des Kosinus, betrachte $x_n = \frac{\pi n}{2}$\\
Lösung: $f(x) = 0$ und nicht gleichmäßig Konvergenz.

Für $t > 0$ sei $g_t : \mathbb{R} \to \mathbb{R}$ mit
\begin{displaymath}
  g_t(x) := \frac{\sqrt{t}}{\pi} \cdot \frac{1}{1 + tx2}.
\end{displaymath}
\begin{enumerate}
  \item Konvergiert die Funktionsfolge $\left(g_{\frac{1}{n}}\right)_{n \in \mathbb{N}}$ punktweise und/oder gleichmäßig?
  Bestimme die Grenzfunktion (falls vorhanden).
  \item Konvergiert die Funktionsfolge $\left(g_n\right)_{n \in \mathbb{N}}$ punktweise und/oder gleichmäßig?
  Bestimme die Grenzfunktion (falls vorhanden).
\end{enumerate}

Lösung:
\begin{enumerate}
  \item Sei $f := 0$.
  Dann gilt
  \begin{displaymath}
    |g_{\frac{1}{n}} - f| \leq \frac{1}{\pi} \cdot \frac{1}{\sqrt{n}} \cdot \frac{1}{1 + 0} \to 0 \quad (n \to \infty).
  \end{displaymath}
  Die Abschätzung ist unabhängig von $x$, also liegt gleichmäßige Konvergenz gegen $f$ vor.
  Mit der gleichmäßigen Konvergenz folgt die punktweise Konvergenz.
  
  \item Betrachte $x = 0$.
  \begin{displaymath}
    g_n(0) = \frac{\sqrt{n}}{\pi} \cdot \frac{1}{1 + n \cdot 0} = \frac{1}{\pi} \cdot \sqrt{n} \to \infty \quad (n \to \infty),
  \end{displaymath}
  also liegt keine punktweise Konvergenz vor.
  Damit auch keine gleichmäßige Konvergenz.
\end{enumerate}

\subsection{Wahr oder Falsch?}
Für alle $n \in \mathbb{N}$ seien $f_n : [0, \infty) \to \mathbb{R}$ stetige Funktionen und es gelte $f_n \xrightarrow{n \to \infty} f$ gleichmäßig auf $[0,\infty)$. Dann ist $f$ stetig.\\
Wahr.\\
Satz aus der VL.

Wenn $x_n$ divergiert, so divergiert auch $(f(x_n))_{n \in \mathbb{N}}$.\\
Falsch.\\
Betrachte $f(x) = \frac{1}{x}$.

Sei $I \subseteq \mathbb{R}$ ein Intervall, seien $f_n, f, g : I \to \mathbb{R}$ Funktionen und sei $g$ beschränkt.
Falls $f_n \to f$ gleichmäßig auf $I$ für $n \to \infty$, dann gilt $f_n \cdot g \to f \cdot g$ gleichmäßig auf $I$ für $n \to \infty$.\\
Wahr.